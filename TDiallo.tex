\documentclass[12pt]{article}
\usepackage{stmaryrd}
\usepackage{graphicx}
\usepackage[utf8]{inputenc}

\usepackage[french]{babel}
\usepackage[T1]{fontenc}
%\usepackage{hyperref}
\usepackage[colorlinks=true, linkcolor=blue, urlcolor=blue, citecolor=blue]{hyperref}
\usepackage{verbatim}

\usepackage{color, soul}

\usepackage{pgfplots}
\pgfplotsset{compat=1.15}
\usepackage{mathrsfs}

\usepackage{amsmath}
\usepackage{amsfonts}
\usepackage{amssymb}
\usepackage{tkz-tab}

\usepackage{tikz}
\usetikzlibrary{arrows, shapes.geometric, fit}


\usepackage[margin=2cm]{geometry}
\usepackage{eso-pic}         % Pour ajouter des éléments en arrière-plan

% Commande pour ajouter du texte en arrière-plan
\AddToShipoutPicture{
    \AtTextCenter{%
        \makebox[0pt]{\rotatebox{45}{\textcolor[gray]{0.9}{\fontsize{5cm}{5cm}\selectfont Pathé BA}}}
    }
}

\begin{document}

\begin{minipage}{0.8\textwidth}
	Talla                        
\end{minipage}
\begin{minipage}{0.8\textwidth}
	Diallo 
\end{minipage}

\begin{center}
\textbf{{\underline{\textcolor{green}{Correction}}}}
\end{center}
\section*{\textcolor{green}{\underline{Exercice 1}:}}

Voici la répartition des salaires annuels (en milliers d'euros) dans une entreprise.

\begin{tabular}{|c|c|c|c|c|c|}
  \hline
  \textbf{Salaire} & [10;20] & [20;30] & [30;40] & [40;50] & [50;60]\\
  \hline
    \textbf{Effectif}& 100 & 60 & 20 & 10 & 10\\
  \hline
\end{tabular}

\begin{enumerate}
  \item[a)] Déterminer le salaire moyen.
  \item[b)] Déterminer un salaire médian.
\end{enumerate}



\end{document}