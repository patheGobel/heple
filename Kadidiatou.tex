\documentclass[12pt]{article}
\usepackage{stmaryrd}
\usepackage{graphicx}
\usepackage[utf8]{inputenc}

\usepackage[french]{babel}
\usepackage[T1]{fontenc}
\usepackage{hyperref}
\usepackage{verbatim}

\usepackage{color, soul}

\usepackage{pgfplots}
\pgfplotsset{compat=1.15}
\usepackage{mathrsfs}

\usepackage{amsmath}
\usepackage{amsfonts}
\usepackage{amssymb}
\usepackage{tkz-tab}

\usepackage{tikz}
\usetikzlibrary{arrows, shapes.geometric, fit}

\usepackage[margin=2cm]{geometry}

\begin{document}

\begin{minipage}{0.8\textwidth}
    Kadidiatou                          
\end{minipage}
\begin{minipage}{0.8\textwidth}
   DIALLO                          
\end{minipage}
\section*{Exercice 1 (06 points)}
\textbf{NB : Les résultats en cm seront donnés à $10^{-2}$ près.}

Un forgeron tape sur une pièce métallique d'épaisseur 1 cm. À chaque coup, l'épaisseur du métal diminue de 1 \%.

On suppose que le temps entre deux coups consécutifs est de 6 secondes.

On désigne par $U_0$ l'épaisseur initiale et par $U_n$ l'épaisseur après $n$ coups (n un entier naturel).

\textbf{Formule :}

La relation de récurrence pour $U_n$ est :

\[ U_{n+1} = U_n \times 0.99 \]

La formule générale pour $U_n$, l'épaisseur après $n$ coups, est donc :

\[ U_n = U_0 \times (0.99)^n \]

\begin{enumerate}
    \item 
    \begin{enumerate}
        \item Calculer $U_1$, $U_2$ et $U_3$:
        \[
        U_1 = U_0 \times (1 - 0.01) = 1 \times 0.99 = 0.99 \text{ cm}
        \]
        \[
        U_2 = U_1 \times (1 - 0.01) = 0.99 \times 0.99 = 0.9801 \text{ cm}
        \]
        \[
        U_3 = U_2 \times (1 - 0.01) = 0.9801 \times 0.99 = 0.970299 \text{ cm}
        \]
        \item Établir une relation entre $U_{n+1}$ et $U_n$:
        \[
        U_{n+1} = U_n \times 0.99
        \]
        En déduire la nature de la suite $(U_n)$:
        \\
        La suite $(U_n)$ est une suite géométrique de raison $r = 0.99$.
    \end{enumerate}

    \item 
    \begin{enumerate}
        \item Exprimer $U_n$ en fonction de $n$:
        \[
        U_n = U_0 \times (0.99)^n
        \]
        \[
        U_n = 1 \times (0.99)^n = (0.99)^n
        \]
        \item En déduire l'épaisseur de la pièce juste avant le 11ème coup:
        \[
        U_{10} = (0.99)^{10} \approx 0.9044 \text{ cm}
        \]
    \end{enumerate}

    \item 
    On considère la pièce finie lorsque son épaisseur est inférieure ou égale à 0.25 cm.
    \begin{enumerate}
        \item Combien de coups seront-ils nécessaires pour terminer la pièce?
        \[
        (0.99)^n \leq 0.25
        \]
        En prenant le logarithme des deux côtés:
        \[
        n \log(0.99) \leq \log(0.25)
        \]
        \[
        n \geq \frac{\log(0.25)}{\log(0.99)} \approx \frac{-0.6021}{-0.0043} \approx 140
        \]

        \item Quel est le temps minimal nécessaire pour terminer la pièce?
        \[
        T = n \times 6 \text{ secondes}
        \]
        \[
        T = 140 \times 6 \approx 840 \text{ secondes}
        \]
    \end{enumerate}
\end{enumerate}

\end{document}
