\documentclass[12pt]{article}
\usepackage{stmaryrd}
\usepackage{graphicx}
\usepackage[utf8]{inputenc}

\usepackage[french]{babel}
\usepackage[T1]{fontenc}
%\usepackage{hyperref}
\usepackage[colorlinks=true, linkcolor=blue, urlcolor=blue, citecolor=blue]{hyperref}
\usepackage{verbatim}

\usepackage{color, soul}

\usepackage{pgfplots}
\pgfplotsset{compat=1.15}
\usepackage{mathrsfs}

\usepackage{amsmath}
\usepackage{amsfonts}
\usepackage{amssymb}
\usepackage{tkz-tab}

\usepackage{tikz}
\usetikzlibrary{arrows, shapes.geometric, fit}


\usepackage[margin=2cm]{geometry}
\usepackage{eso-pic}         % Pour ajouter des éléments en arrière-plan

% Commande pour ajouter du texte en arrière-plan
\AddToShipoutPicture{
    \AtTextCenter{%
        \makebox[0pt]{\rotatebox{45}{\textcolor[gray]{0.9}{\fontsize{5cm}{5cm}\selectfont Pathé BA}}}
    }
}

\begin{document}

\begin{minipage}{0.8\textwidth}
	Talla                        
\end{minipage}
\begin{minipage}{0.8\textwidth}
	Diallo 
\end{minipage}

\begin{center}
\textbf{{\underline{\textcolor{green}{Correction}}}}
\end{center}
\section*{\textcolor{green}{\underline{Exercice 1}:}}
Traduit par une formule mathématique utilisant les quantificateurs les affirmations ci-dessous:

($f$ est une fonction numérique d'une variable réelle, $\ell$ un réel)
\begin{enumerate}
\item \[\lim_{x \to -\infty} f(x) = l\]

\item \[\lim_{x \to +\infty} f(x) = +\infty\]

\item \[\lim_{x \to -\infty} f(x) = +\infty\]

\item \[\lim_{x \to -2} \ln(2x+5) = 0\]

%\item \[\lim_{x \to 2} \frac{x^{2}-3x+5}{(x-2)^{2}} = +\infty\]
\end{enumerate}
\section*{\textcolor{green}{\underline{Correction Exercice 1}:}}
1. 
\[
\forall \epsilon > 0, \exists M \in \mathbb{R}, \text{ tel que } \forall x < M, |f(x) - l| < \epsilon.
\]
Cette expression signifie que, pour tout $\epsilon$ positif, il existe un réel $M$ tel que pour tout $x$ inférieur à $M$, la différence entre $f(x)$ et ll est inférieure à$\epsilon$ , c'est-à-dire que $f(x)$ se rapproche de $\ell$ lorsque $x$ tend vers $-\infty$.

2.
\[
\forall A > 0, \exists M \in \mathbb{R}, \text{ tel que } \forall x > M, f(x) > A.
\]


3. 
\[
\forall A > 0, \exists M \in \mathbb{R}, \text{ tel que } \forall x < M, f(x) > A.
\]


4. 
\[
\forall \epsilon > 0, \exists \delta > 0, \text{ tel que } 0 < |x + 2| < \delta \implies |\ln(2x+5)| < \epsilon.
\]


5. 

\[
\forall A > 0, \exists \delta > 0, \text{ tel que } 0 < |x - 2| < \delta \implies \frac{x^{2}-3x+5}{(x-2)^{2}} > A.
\]

\section*{Exercice 2}

1. \text{En utilisant la définition, montrer que } 
\[\lim_{x \to 5} \sqrt{x} = \sqrt{5}\].

2. \text{Utiliser un encadrement pour déterminer la limite en } $+\infty$ \text{ de la fonction } f \text{ définie par } $f(x) = \frac{\mathrm{E}(x)}{x}$.

\section*{\textcolor{green}{\underline{Correction Exercice 2}:}}

1. \textbf{En utilisant la définition, montrer que}
\[
\lim_{x \to 5} \sqrt{x} = \sqrt{5}.
\]
\textbf{Solution} : Utilisons la définition formelle de la limite. Nous devons montrer que :
\[
\forall \epsilon > 0, \exists \delta > 0, \text{ tel que } 0 < |x - 5| < \delta \implies |\sqrt{x} - \sqrt{5}| < \epsilon.
\]
En d'autres termes, pour tout \( \epsilon > 0 \), il existe un \( \delta > 0 \) tel que si \( x \) est suffisamment proche de 5 (c'est-à-dire que \( |x - 5| < \delta \)), alors \( \sqrt{x} \) est proche de \( \sqrt{5} \) (c'est-à-dire que \( |\sqrt{x} - \sqrt{5}| < \epsilon \)).

Pour déterminer \( \delta \), on commence par manipuler l'inégalité \( |\sqrt{x} - \sqrt{5}| < \epsilon \) :

\[
|\sqrt{x} - \sqrt{5}| = \frac{|x - 5|}{\sqrt{x} + \sqrt{5}}.
\]

On cherche donc \( \delta \) tel que :
\[
\frac{|x - 5|}{\sqrt{x} + \sqrt{5}} < \epsilon.
\]

Puisque \( \sqrt{x} \) est proche de \( \sqrt{5} \) lorsque \( x \to 5 \), on peut majorer \( \sqrt{x} + \sqrt{5} \) par une constante, et ainsi choisir \( \delta \) en fonction de \( \epsilon \).

2. \textbf{Utiliser un encadrement pour déterminer la limite en } $+\infty$ \textbf{ de la fonction } f \textbf{ définie par } $f(x) = \frac{E(x)}{x}$.

\textbf{Solution} : Si \( E(x) \) désigne la partie entière de \( x \), on peut encadrer \( f(x) \) de la manière suivante :
\[
\frac{x-1}{x} \leq \frac{E(x)}{x} \leq \frac{x}{x} = 1.
\]

Ainsi, on a l'encadrement :
\[
1 - \frac{1}{x} \leq f(x) \leq 1.
\]

En passant à la limite lorsque \(x \to +\infty\), on obtient :
\[
\lim_{x \to +\infty} \left( 1 - \frac{1}{x} \right) = 1 \quad \text{et} \quad \lim_{x \to +\infty} 1 = 1.
\]

Par le théorème des gendarmes, on en déduit que :
\[
\lim_{x \to +\infty} f(x) = 1.
\]

\section*{Exercice 3}

1. \text{Quand dit-on qu'une fonction } \( f \) \text{ est minorée sur un intervalle } \( I \)\text{ ?}

2. \text{Quand dit-on qu'une fonction } \( f \) \text{ est continue sur un intervalle du type } \([a, b]\) \text{ avec } \(a, b \in \mathbb{R}\) \text{ ?}

3. \text{Donner la proposition exprimant la propriété appelée "Inégalité des accroissements finis".}

4. \text{Donner la proposition exprimant la propriété appelée "Règle de l'Hospital".}

5. \text{Justifier que } \[ \lim_{{x \to 0}} \frac{e^{x}-1}{\sin x} = 1 \].

6. \text{Montrer que si } \( f \) \text{ et } \( g \) \text{ sont deux fonctions croissantes, alors la fonction } \( f + g \)\\ \text{ est une fonction croissante.}

7.\text{Peut-on prolonger par continuité en } \( x_0 = 1\) \text{ la fonction définie par } \(f(x) = \frac{x^2 - 1}{x^3 - 1}\)\text{ ? Justifier.}
\section*{\textcolor{green}{\underline{Correction Exercice 3}:}}
1. On dit qu'une fonction \( f \) est minorée sur un intervalle \( I \) s'il existe un réel \( m \in \mathbb{R} \) tel que pour tout \( x \in I \), on ait :
\[
f(x) \geq m.
\]
Autrement dit, \( m \) est une borne inférieure de \( f \) sur \( I \).


2.On dit qu'une fonction \( f \) est continue sur un intervalle \([a, b]\) si elle est continue en tout point de cet intervalle.

Autrement dit, pour tout \( c \in [a, b] \), la fonction \( f \) satisfait :
\[
\lim_{x \to c} f(x) = f(c).
\]

3.Soit \( f \) une fonction continue sur un intervalle fermé \([a, b]\) et dérivable sur l'intervalle ouvert \( ]a, b[ \). Alors, il existe un réel \( c \in [a, b] \) tel que :
\[
f'(c) = \frac{f(b) - f(a)}{b - a}.
\]

3.Soit \( f \) et \( g \) deux fonctions réelles définies sur un intervalle \( I \) contenant un point \( a \) (sauf peut-être en \( a \)) et telles que \( g(x) \neq 0 \) pour tout \( x \in I \) sauf peut-être en \( a \). Supposons que :

- \( \lim_{x \to a} f(x) = 0 \) et \( \lim_{x \to a} g(x) = 0 \) (forme indéterminée \( \frac{0}{0} \)), 

ou

- \( \lim_{x \to a} f(x) = \pm \infty \) et \( \lim_{x \to a} g(x) = \pm \infty \) (forme indéterminée \( \frac{\infty}{\infty} \)).

Si \( f \) et \( g \) sont dérivables sur \( I \) et que \( g'(x) \neq 0 \) pour tout \( x \) dans \( I \) (sauf peut-être en \( a \)), alors :
\[
\lim_{x \to a} \frac{f(x)}{g(x)} = \lim_{x \to a} \frac{f'(x)}{g'(x)},
\]

5. \textbf{Justifions que}
\[
\lim_{x \to 0} \frac{e^{x}-1}{\sin x} = 1.
\]

\textbf{Justification} : Pour démontrer cette limite, nous pouvons utiliser la règle de l'Hôpital, car à \( x = 0 \), nous avons la forme indéterminée \( \frac{0}{0} \).

Calculons les limites des fonctions au numérateur et au dénominateur :

- Pour le numérateur : \( \lim_{x \to 0} (e^{x} - 1) = e^{0} - 1 = 0 \).

- Pour le dénominateur : \( \lim_{x \to 0} \sin x = \sin(0) = 0 \).

Nous appliquons donc la règle de l'Hôpital :
\[
\lim_{x \to 0} \frac{e^{x}-1}{\sin x} = \lim_{x \to 0} \frac{(e^{x})'}{(\sin x)'} = \lim_{x \to 0} \frac{e^{x}}{\cos x}.
\]

À ce stade, nous évaluons la nouvelle limite :
\[
\lim_{x \to 0} \frac{e^{x}}{\cos x} = \frac{e^{0}}{\cos(0)} = \frac{1}{1} = 1.
\]

Ainsi, nous avons justifié que :
\[
\lim_{x \to 0} \frac{e^{x}-1}{\sin x} = 1.
\]

6. \textbf{Montrons que si } \( f \) \text{ et } \( g \) \text{ sont deux fonctions croissantes, alors la fonction } \( f + g \)\\ \text{ est une fonction croissante.}

\textbf{Démonstration} : Soit \( f \) et \( g \) deux fonctions croissantes sur un intervalle \( I \).

Cela signifie que pour tous \( x_1, x_2 \in I \) tels que \( x_1 < x_2 \), nous avons :
\[
f(x_1) \leq f(x_2) \quad \text{et} \quad g(x_1) \leq g(x_2).
\]

Nous voulons montrer que \( f + g \) est croissante sur \( I \). Pour cela, considérons deux points \( x_1 \) et \( x_2 \) dans \( I \) avec \( x_1 < x_2 \). Nous avons alors :
\[
f(x_1) \leq f(x_2) \quad \text{et} \quad g(x_1) \leq g(x_2).
\]

En additionnant ces inégalités, nous obtenons :
\[
f(x_1) + g(x_1) \leq f(x_2) + g(x_2).
\]

Ainsi, nous avons :
\[
(f + g)(x_1) \leq (f + g)(x_2).
\]

Cela montre que \( f + g \) est croissante. Par conséquent, nous concluons que si \( f \) et \( g \) sont deux fonctions croissantes, alors la fonction \( f + g \) est également croissante.

7. Pour déterminer si nous pouvons prolonger la fonction \( f \) par continuité en \( x_0 = 1 \), nous devons d'abord examiner la fonction au point \( x = 1 \). 

Calculons la valeur de la fonction \( f(x) \) à ce point :
\[
f(1) = \frac{1^2 - 1}{1^3 - 1} = \frac{0}{0}.
\]
La fonction n'est pas définie en \( x = 1 \) car nous avons une forme indéterminée \( \frac{0}{0} \).

Pour étudier la limite de \( f(x) \) lorsque \( x \) tend vers 1, nous allons simplifier l'expression :
\[
f(x) = \frac{x^2 - 1}{x^3 - 1} = \frac{(x - 1)(x + 1)}{(x - 1)(x^2 + x + 1)} \quad \text{pour } x \neq 1.
\]
Nous pouvons simplifier \( f(x) \) :
\[
f(x) = \frac{x + 1}{x^2 + x + 1} \quad \text{pour } x \neq 1.
\]

Calculons la limite lorsque \( x \) tend vers 1 :
\[
\lim_{x \to 1} f(x) = \lim_{x \to 1} \frac{x + 1}{x^2 + x + 1} = \frac{1 + 1}{1^2 + 1 + 1} = \frac{2}{3}.
\]

Ainsi, la limite de \( f(x) \) lorsque \( x \to 1 \) existe et vaut \( \frac{2}{3} \).

Nous pouvons donc définir \( f(1) = \frac{2}{3} \) pour prolonger la fonction par continuité. Ainsi, nous pouvons conclure que :

Oui, nous pouvons prolonger \( f \) par continuité en \( x_0 = 1 \) en posant \( f(1) = \frac{2}{3} \).

\end{document}