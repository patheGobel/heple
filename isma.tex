\documentclass[12pt]{article}
\usepackage{stmaryrd}
\usepackage{graphicx}
\usepackage[utf8]{inputenc}

\usepackage[french]{babel}
\usepackage[T1]{fontenc}
\usepackage{hyperref}
\usepackage{verbatim}

\usepackage{color, soul}

\usepackage{pgfplots}
\pgfplotsset{compat=1.15}
\usepackage{mathrsfs}

\usepackage{amsmath}
\usepackage{amsfonts}
\usepackage{amssymb}
\usepackage{tkz-tab}

\usepackage{tikz}
\usetikzlibrary{arrows, shapes.geometric, fit}

\usepackage[margin=2cm]{geometry}

\begin{document}

\begin{minipage}{0.8\textwidth}
    Mamadou Demba                           
\end{minipage}
\begin{minipage}{0.8\textwidth}
   DIALLO                          
\end{minipage}
\section*{Exercice }
Pour un entier naturel \( n \geq 2 \), la fonction \( f_n \) est définie par :
\[ f_n(x) = \frac{1 + n \ln x}{x^2} \]
pour \( x > 0 \).

\section*{1. Calcul de la dérivée \( f'_n(x) \)}
Appliquons la règle de dérivation d'un quotient :

\[ f'_n(x) = \frac{(1 + n \ln x)' \cdot x^2 - (1 + n \ln x) \cdot (x^2)'}{(x^2)^2} \]

\[ f'_n(x) = \frac{\left( \frac{n}{x} \right) \cdot x^2 - (1 + n \ln x) \cdot 2x}{x^4} \]

\[ f'_n(x) = \frac{n x - 2x (1 + n \ln x)}{x^4} \]

\[ f'_n(x) = \frac{nx - 2x - 2nx \ln x}{x^4} \]

\[ f'_n(x) = \frac{x (n - 2 - 2n \ln x)}{x^3} \]

Ainsi, la dérivée de \( f_n(x) \) est :

\[ f'_n(x) = \frac{n - 2 - 2n \ln x}{x^3} \]

\section*{2. Étude du signe de \( f'_n(x) \)}
\begin{center}
	\textbf{\textcolor{red}{Autre Approche}}
\end{center}

Le signe de \( f'_n(x) \), dépend du numérateur \( n - 2 - 2n \ln x \) car \( \forall x>, x^{3}>0 \).

\[\text{Supposons que }  n - 2 - 2n \ln x >0 \text{ donc : }\]

\[ n - 2 - 2n \ln x > 0 \implies 2n \ln x < n - 2 \implies \ln x < \frac{n - 2}{2n} \implies x < e^{\frac{n - 2}{2n}} \implies x\in \left]0,e^{\frac{n - 2}{2n}}\right[ \] 

\[\text{Donc si } x\in \left]0,e^{\frac{n - 2}{2n}}\right[ \text{alors } n - 2 - 2n \ln x >0 \text{ par conséquent, } f'_n(x)>0. \]

\[\text{si } x\notin \left]0,e^{\frac{n - 2}{2n}}\right[ \text{ c'est-à-dire } x\in \left]e^{\frac{n - 2}{2n}},+\infty\right[ \text{alors } n - 2 - 2n \ln x < 0 \text{ par conséquent, } f'_n(x)>0. \]

\section*{Conclusion sur le signe de \( f'_n(x) \)}
\begin{itemize}
\item[-] \( \forall x\in \left] 0, e^{\frac{n - 2}{2n}}\right[  \) \( f'_n(x) > 0 \) donc $f_n(x)$ est croissant

\item[-] \( \forall x\in  \left] e^{\frac{n - 2}{2n}}, +\infty \right[ \) \( f'_n(x) < 0 \) donc $f_n(x)$ est croissant.
\end{itemize} 

\section*{Limites aux bornes de Df}

\subsection*{En \( 0^+ \)}

\begin{align*}
\lim_{x \to 0^{+}}f_n(x)=\lim_{x \to 0^{+}}\frac{1 + n \ln x}{x^2}:
\begin{cases}
\lim_{x \to 0^{+}}1 + n \ln x=-\infty\\
\lim_{x \to 0^{+}}x^2=0^{+}
\end{cases}
\text{ par quotient, }-\infty
\end{align*}

 \[\text{ Donc, } \textcolor{green}{\boxed{\lim_{x \to 0^+} f_n(x) = -\infty}}.\]

\subsection*{En \(  +\infty \)}

\begin{align*}
\lim_{x \to +\infty }f_n(x)=\lim_{x \to +\infty }\frac{1 + n \ln x}{x^2}:
\begin{cases}
\lim_{x \to +\infty}1 + n \ln x=+\infty\\
\lim_{x \to +\infty}x^2=+\infty
\end{cases}
\text{ par quotient, FI}\\
\end{align*}
\text{ Levons l'indétermination}
\begin{align*}
\lim_{x \to +\infty }\frac{1 + n \ln x}{x^2}=\lim_{x \to +\infty }\frac{1}{x^2}+\frac{ n \ln x}{x^2}=\lim_{x \to +\infty }n\frac{ 1}{x} \times \frac{ \ln x}{x}=0
\end{align*}
\[\text{ Donc, } \textcolor{green}{\boxed{\lim_{x \to +\infty} f_n(x) = 0}}.\]
\section*{Résumé}

\[
\textcolor{green}{\boxed{\lim_{x \to 0^+} f_n(x) = -\infty}}
\]

\[
\textcolor{green}{\boxed{\lim_{x \to +\infty} f_n(x) = 0}}
\]

\begin{tikzpicture}
    % Initialisation du tableau
    \tkzTabInit{$x$/1,$f_{n}'(x)$/1,$f_{n}(x)$/2}{$0$,$e^{\frac{n-2}{2n}}$,$+\infty$}
    % Ligne des variations de f'
    \tkzTabLine{,+,z,-,}
    % Valeurs de f et points critiques
    \tkzTabVar{-/$-\infty$,+/$\frac{n}{2e}e^{\frac{2}{n}}$,-/$0$}
\end{tikzpicture}
\section*{3.}
\begin{enumerate}
\item[(a)] Voir PDF
\\
\begin{figure}[h]
\centering
\includegraphics[width=1.3\textwidth]{c2c3.png}
\caption{Courbe de (Cf)}
\label{fig:monimage}
\end{figure}
\item[(b)] Calculons \( f_{n+1}(x)-f_{n}(x) \).
Pour un entier naturel \( n \geq 2 \), la fonction \( f_n \) est définie par :
\[ f_n(x) = \frac{1 + n \ln x}{x^2} \]

La fonction \( f_{n+1}(x) \) est donc :
\[ f_{n+1}(x) = \frac{1 + (n+1) \ln x}{x^2} \]

Calculons \( f_{n+1}(x) - f_n(x) \) :
\[ f_{n+1}(x) - f_n(x) = \frac{1 + (n+1) \ln x}{x^2} - \frac{1 + n \ln x}{x^2} \]

Mettons les fractions sous un même dénominateur :
\[ f_{n+1}(x) - f_n(x) = \frac{1 + (n+1) \ln x - (1 + n \ln x)}{x^2} \]

Simplifions le numérateur :
\[ 1 + (n+1) \ln x - 1 - n \ln x = (n+1) \ln x - n \ln x = \ln x \]

Ainsi, nous avons :
\[ f_{n+1}(x) - f_n(x) = \frac{\ln x}{x^2} \]

Donc, la différence entre \( f_{n+1}(x) \) et \( f_n(x) \) est :
\[ \textcolor{green}{\boxed{f_{n+1}(x) - f_n(x) = \frac{\ln x}{x^2}}} \]
\item[(c)] \section*{Expliquons comment la courbe \(C_4\) est obtenue à partir de \(C_2\) et \(C_3\)}

Pour obtenir la courbe \( C_4 \) à partir des courbes \( C_2 \) et \( C_3 \), nous devons comprendre la relation entre ces courbes et comment elles sont définies par la fonction \( f_n(x) = \frac{1 + n \ln x}{x^2} \).

\subsection*{Relation entre \(C_2\), \(C_3\), et \(C_4\)}

\subsubsection*{\underline{ Expression de } \( f_n(x) \)}
La fonction \( f_n(x) \) est définie par :
\[ f_n(x) = \frac{1 + n \ln x}{x^2} \]

\subsubsection*{\underline{ Courbes \(C_2\) et \(C_3\)}}
\begin{itemize}
    \item[-] \( C_2 \) est la courbe de \( f_2(x) = \frac{1 + 2 \ln x}{x^2} \).
    \item[-] \( C_3 \) est la courbe de \( f_3(x) = \frac{1 + 3 \ln x}{x^2} \).
\end{itemize}

\subsubsection*{\underline{ Courbe \(C_4\)}}
\begin{itemize}
    \item[-] \( C_4 \) sera la courbe de \( f_4(x) = \frac{1 + 4 \ln x}{x^2} \).
\end{itemize}

\subsection*{Méthode pour obtenir \(C_4\)}

Pour obtenir \( C_4 \) à partir de \( C_2 \) et \( C_3 \), nous devons comprendre comment les termes supplémentaires \( 2 \ln x \) et \( 3 \ln x \) affectent la courbe. Ensuite, nous pouvons utiliser cette compréhension pour construire \( C_4 \).

\subsubsection*{\underline{ Analyser les termes supplémentaires }}
\begin{itemize}
    \item[-] \( C_2 \) et \( C_3 \) diffèrent par un terme \(\frac{\ln x}{x^2}\).
    \item[-] De même, \( C_4 \) diffèrera de \( C_3 \) par un terme \(\frac{\ln x}{x^2}\).
\end{itemize}

\subsubsection*{\underline{  Comprendre l'effet des termes supplémentaires }}
\begin{itemize}
    \item[-] La courbe \( C_4 \) peut être vue comme une transformation de \( C_3 \), ajoutant encore un terme \(\frac{\ln x}{x^2}\).
\end{itemize}

\subsubsection*{\underline{ Utiliser la différence }}
\begin{itemize}
    \item[-] Nous avons calculé précédemment que \( f_{n+1}(x) - f_n(x) = \frac{\ln x}{x^2} \).
    \item[-] Pour obtenir \( C_4 \) à partir de \( C_2 \) et \( C_3 \), on peut superposer \( C_3 \) sur \( C_2 \) et ajouter le terme supplémentaire \(\frac{\ln x}{x^2}\) à chaque point de \( C_3 \).
\end{itemize}

\end{enumerate}
\section*{4. On considère la suite de terme général $A_{n}=\int_{1}^{e}f_{n}(x)dx$}
\begin{itemize}
\item[(a)] Calculons $A_{n}$ Par intégration par parties et interprétons géométriquement.

Considérons la suite de terme général \( A_n = \int_{1}^{e} f_n(x) \, dx \) avec \( f_n(x) = \frac{1 + n \ln x}{x^2} \).

Intégration par parties :

Posons 
\[ u = 1 + n \ln x \implies u' = \frac{n}{x} \]

Nous avons 
\[ v' = \frac{1}{x^2} \implies v = -\frac{1}{x} \]

L'intégration par parties donne :
\[ \int u \, v' = uv - \int v \, u' \]

Appliquons cette méthode à \( A_n \) :
\begin{align*}
A_n &=  -\frac{1 + n \ln x}{x}+\int_{1}^{e}\frac{n}{x^{2}}\\
&=-\left[ \frac{1 + n \ln x}{x}\right]_{1}^{e} -\left[ \frac{n}{x}\right] _{1}^{e}\\
&=-\left( \frac{1 + n \ln e}{e}-\frac{1 + n \ln 1}{1}\right) -\left( \frac{n}{e}-\frac{n}{1}\right)\\
&=-\left( \frac{1 + n-e}{e}\right) -\left( \frac{n-ne}{e}\right)\\
&=\frac{-1 - n+e}{e} + \frac{-n+ne}{e}\\
&=\frac{-1 - 2n+e+ne}{e}\\
\end{align*}
\[\textcolor{green}{\boxed{A_n=\frac{-1+e-n(2+e)}{e}}}\]



\end{itemize}
\section*{5. Dans la suite, on considère que $n\geq 3$.}
\begin{itemize}
\item[(a)] Montrer que $ e^{\frac{n-2}{2n}} > 0 $ et $f_{n}(e^{\frac{n-2}{2n}}) > 0$.
\item[(b)] En déduire que l'équation $f_{n}(x)=1$ n'admet pas de solution sur l'intervalle $]1,e^{\frac{n-2}{2n}}[$
\end{itemize}
\end{document}