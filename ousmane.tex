\documentclass[12pt,a4paper]{article}
\usepackage{amsmath,amssymb,mathrsfs,tikz,times,pifont}
\usepackage{enumitem}
\newcommand\circitem[1]{%
\tikz[baseline=(char.base)]{
\node[circle,draw=gray, fill=red!55,
minimum size=1.2em,inner sep=0] (char) {#1};}}
\newcommand\boxitem[1]{%
\tikz[baseline=(char.base)]{
\node[fill=cyan,
minimum size=1.2em,inner sep=0] (char) {#1};}}
\setlist[enumerate,1]{label=\protect\circitem{\arabic*}}
\setlist[enumerate,2]{label=\protect\boxitem{\alph*}}
%%%::::::by chnini ameur :::::::%%%
\everymath{\displaystyle}
\usepackage[left=1cm,right=1cm,top=1cm,bottom=1.7cm]{geometry}
\usepackage{array,multirow}
\usepackage[most]{tcolorbox}
\usepackage{varwidth}
\tcbuselibrary{skins,hooks}
\usetikzlibrary{patterns}
%%%::::::by chnini ameur :::::::%%%
\newtcolorbox{exa}[2][]{enhanced,breakable,before skip=2mm,after skip=5mm,
colback=yellow!20!white,colframe=black!20!blue,boxrule=0.5mm,
attach boxed title to top left ={xshift=0.6cm,yshift*=1mm-\tcboxedtitleheight},
fonttitle=\bfseries,
title={#2},#1,
% varwidth boxed title*=-3cm,
boxed title style={frame code={
\path[fill=tcbcolback!30!black]
([yshift=-1mm,xshift=-1mm]frame.north west)
arc[start angle=0,end angle=180,radius=1mm]
([yshift=-1mm,xshift=1mm]frame.north east)
arc[start angle=180,end angle=0,radius=1mm];
\path[left color=tcbcolback!60!black,right color = tcbcolback!60!black,
middle color = tcbcolback!80!black]
([xshift=-2mm]frame.north west) -- ([xshift=2mm]frame.north east)
[rounded corners=1mm]-- ([xshift=1mm,yshift=-1mm]frame.north east)
-- (frame.south east) -- (frame.south west)
-- ([xshift=-1mm,yshift=-1mm]frame.north west)
[sharp corners]-- cycle;
},interior engine=empty,
},interior style={top color=yellow!5}}
%%%%%%%%%%%%%%%%%%%%%%%

\usepackage{fancyhdr}
\usepackage{eso-pic}         % Pour ajouter des éléments en arrière-plan
% Commande pour ajouter du texte en arrière-plan
\AddToShipoutPicture{
    \AtTextCenter{%
        \makebox[0pt]{\rotatebox{80}{\textcolor[gray]{0.7}{\fontsize{5cm}{5cm}\selectfont PGB}}}
    }
}
\usepackage{lastpage}
\fancyhf{}
\pagestyle{fancy}
\renewcommand{\footrulewidth}{1pt}
\renewcommand{\headrulewidth}{0pt}
\renewcommand{\footruleskip}{10pt}
\fancyfoot[R]{
\color{blue}\ding{45}\ \textbf{2024}
}
\fancyfoot[L]{
\color{blue}\ding{45}\ \textbf{Prof:M. Diallo}
}
\cfoot{\bf
\thepage /
\pageref{LastPage}}
\begin{document}
\renewcommand{\arraystretch}{1.5}
\renewcommand{\arrayrulewidth}{1.2pt}
\begin{tikzpicture}[overlay,remember picture]
\node[draw=blue,line width=1.2pt,fill=purple,text=blue,inner sep=3mm,rounded corners,pattern=dots]at ([yshift=-2.5cm]current page.north) {\begingroup\setlength{\fboxsep}{0pt}\colorbox{white}{\begin{tabular}{|*1{>{\centering \arraybackslash}p{0.28\textwidth}} |*2{>{\centering \arraybackslash}p{0.2\textwidth}|} *1{>{\centering \arraybackslash}p{0.19\textwidth}|} }
\hline
\multicolumn{3}{|c|}{$\diamond$$\diamond$$\diamond$\ \textbf{Lycée de Dindéfélo}\ $\diamond$$\diamond$$\diamond$ }& \textbf{A.S. : 2024/2025} \\ \hline
\textbf{Matière: Mathématiques}& \textbf{Niveau : T}\textbf{L} &\textbf{Date: 09/12/2024} & \textbf{Durée : 2 heures} \\ \hline
\multicolumn{4}{|c|}{\parbox[c]{10cm}{\begin{center}
\textbf{{\Large\sffamily Devoir n$ ^{\circ} $ 1 Du 1$ ^\text{\bf er} $ Semestre}}
\end{center}}} \\ \hline
\end{tabular}}\endgroup};
\end{tikzpicture}
\vspace{3cm}

\section*{\underline{Exercice 1 :} $7$ points}

Soient \( f \) et \( g \) deux fonctions définies par :
\[ f(x) = x^2 - x - 14 \quad \text{et} \quad g(x) = \sqrt{-2x + 1}. \]

\begin{enumerate}
    \item Déterminer le domaine de définition \( D_f \) de \( f \) et \( D_g \) de \( g \). \hfill (1pt + 1pt)
    \item Déterminer \( f \circ g(x) \) et \( g \circ f(x) \). \hfill (1pt + 1pt)
    \item Calculer \( f \circ g(-4) \) et \( g \circ f(2) \). \hfill (0.5pt + 0.5pt)
    \item On considère les deux applications \( h(x) = 2x + 3 \) et \( p(x) = 3x - 4 \).
        \begin{enumerate}
            \item Vérifier que \( h \circ p(2) = 7 \). \hfill (0.5pt)
            \item Calculer \( p \circ h(-\frac{3}{2}) \). \hfill (0.5pt)
        \end{enumerate}
    \item On donne la fonction \( q \) définie par \( q(x) = \frac{2x^2 - 5x + 3}{x - 2} \). Pour tout \( x \neq 2 \), déterminer les réels \( a, b \) et \( c \) tels que \( q(x) = ax + b + \frac{c}{x - 2} \). \hfill (1pt)
\end{enumerate}

\section*{\underline{Exercice 1 :} $13$ points}

Soit le polynôme \( p(x) = ax^3 + 17x^2 - 9x - 5 \) où \( a \) est un nombre réel.

\begin{enumerate}
    \item Trouver le réel \( a \) tel que \( p(1) = 0 \). \hfill (1pt)
    \item On donne le polynôme \( p(x) = 3x^3 + 17x^2 - 9x - 5 \).
        \begin{enumerate}
            \item Montrer que \(-1\) est une racine de \( p \). \hfill (1pt)
            \item Montrer que \( p(x) = (x + 1)q(x) \) où \( q(x) \) est un polynôme à préciser. \hfill (1.5pt)
            \item Factoriser complètement \( p(x) \). \hfill (1.5pt)
        \end{enumerate}
    \item On pose \( p(x) = (3x - 1)(x + 5)(x + 1) \). Résoudre dans \( \mathbb{R} \) :
        \begin{enumerate}
            \item \( p(x) = 0 \) ; \hfill (1pt)
            \item \( p(x) \leq 0 \). \hfill (2pts)
        \end{enumerate}
    \item Soit \( f \) la fraction rationnelle définie par \( f(x) = \frac{p(x)}{x^2 - 1} \).
        \begin{enumerate}
            \item Simplifier \( f(x) \). \hfill (1pt)
            \item Déterminer le domaine de définition \( D_f \) de \( f \). \hfill (1pt)
            \item Résoudre dans \( \mathbb{R} \) :
                \begin{itemize}
                    \item \( f(x) = 0 \) ; \hfill (1pt)
                    \item \( f(x) \geq 0 \). \hfill (2pts)
                \end{itemize}
        \end{enumerate}
\end{enumerate}

\end{document}
