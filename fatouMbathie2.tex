\documentclass{article}
\usepackage[utf8]{inputenc}  % Gestion des caractères UTF-8
\usepackage{graphicx}        % Inclusion d'images
\usepackage{geometry}        % Gestion des marges
\usepackage{fancyhdr}        % Personnalisation des en-têtes et pieds de page
\usepackage{amsmath}         % Environnements et commandes mathématiques avancés
\usepackage{amssymb}         % Symboles mathématiques supplémentaires
\usepackage{amsfonts}        % Polices mathématiques supplémentaires
\usepackage{mathtools}       % Extensions et améliorations de amsmath
\usepackage{esint}           % Symboles d'intégrale supplémentaires

\geometry{top=2cm, bottom=2cm, left=2cm, right=2cm}

% Ajuster la hauteur de l'en-tête si nécessaire
\setlength{\headheight}{25pt} 

% Personnalisation de l'en-tête uniquement pour la première page
\fancypagestyle{firstpage}{
    \fancyhead[L]{\includegraphics[width=1.5cm]{flag.png}} % Image du drapeau
    \fancyhead[R]{\raisebox{-0.3cm}{\includegraphics[width=1.5cm]{logo.png}}\hspace{0.1cm}} % Image du logo
    \fancyhead[C]{\hspace{-1.5cm} \textbf{République du Sénégal\\ Un peuple – Un but – Une foi\\ Ministère de l'Éducation Nationale\\ IA Tambacounda}}
    \fancyfoot{} % Pas de pied de page
    \renewcommand{\headrulewidth}{0pt} % Pas de ligne sous l'en-tête
}

\pagestyle{plain} % Style de page par défaut pour les pages suivantes

\begin{document}

% Appliquer le style "firstpage" pour la première page
\thispagestyle{firstpage}

% Lignes sous l'entête
\noindent
\begin{tabular}{@{}p{0.5\textwidth}p{0.5\textwidth}@{}}
    \textbf{ANNÉE SCOLAIRE : 2024/2025} & \raggedleft \textbf{NIVEAU : 4ième}
\end{tabular}

\vspace{0.5cm}

% Contenu centré
\begin{center}
    \textbf{DURÉE : 2H}\\
    \textbf{DEVOIR N°1 DU PREMIER SEMESTRE : ZONE B-B-G-G-T-T}\\
    \textbf{\Large ÉPREUVE DE MATHÉMATIQUES}
\end{center}

\vspace{0.5cm}

\section*{\underline{Exercice 1 : 7,5 points}}

\textbf{Partie A : Réponds par Vrai ou Faux}

\begin{enumerate}
    \item $\frac{1}{3}$ est un nombre décimal \dotfill
    \item $3,14\ldots$ est un nombre rationnel \dotfill
    \item On a $-3^2 = (-3)^2$ \dotfill
    \item Quel que soit le nombre rationnel $A$, $-A$ est négatif \dotfill
    \item Tout nombre qui s'écrit sous la forme $\frac{a}{b}$ est un nombre rationnel \dotfill
    \item Tout nombre décimal est un nombre rationnel \dotfill
\end{enumerate}

\textbf{Partie B : Recopie puis complète les phrases suivantes.}

\begin{enumerate}
    \item Soit $a \in \mathbb{Z}$ et $b \in \mathbb{Z}$, le nombre $\frac{a}{b}$ est \dotfill avec \dotfill
    \item Soit $\frac{a}{b}$ un nombre rationnel et $n$ un nombre entier naturel, on a : 
    $\left(\frac{a}{b}\right)^{-n} = \left(\frac{b}{a}\right)^n$ \dotfill
    \item Si $|a| = |b|$ alors \dotfill
    \item Si $a > b$ et $c$ nombre négatif alors $a \times c$ \dotfill $b \times c$
    \item La valeur absolue d’un nombre rationnel n’est jamais \dotfill
    \item Les rationnel $a$ et $b$ sont inverses si leur \dotfill est égal à \dotfill
    \item L’inverse de $\frac{1}{3}$ est \dotfill et l’opposé de $\frac{132}{11}$ est \dotfill
\end{enumerate}

\section*{Exercice 2 :}

\begin{enumerate}
    \item Complète les pointillés par le nombre qui convient :
    \[
    \frac{9}{7} = \frac{\dots}{-133} \, ; \quad \frac{5}{3} = \frac{45}{\dots} \, ; \quad \frac{\dots}{8} = \frac{36}{32} \, ; \quad \frac{-5}{\dots} = \frac{65}{169}
    \]
    
    \item Rends irréductible les nombres rationnels suivants :
    \[
    \frac{210}{165} \, ; \quad \frac{63}{-18} \, ; \quad \frac{3500}{4900} \, ; \quad \frac{-84}{72}
    \]
        \item[3)] Donne l'écriture scientifique des nombres suivants :
    \[
    A = 0,0009 \quad \quad B = -0,00016 \quad \quad C = 13000 \quad \quad D = -7350
    \]
\end{enumerate}

\section*{Exercice 3}

\begin{enumerate}
    \item Calcule et donne le résultat sous la forme irréductible :
    \[
    A = \frac{-2}{3} \times \frac{27}{25} \times \frac{-5}{9} \quad
    B = -\frac{3}{2} + \frac{1}{4} + \frac{1}{6}
    \]
    \[
    C = \left( -\frac{3}{4} \div \frac{81}{16} \right) \times \frac{27}{-4} \quad
    D = \left( 41 \times \frac{2}{13} \right) \div \left( 1 + \frac{7}{8} \right)
    \]

    \item Écris sous la forme d’une puissance de 10 les expressions suivantes :
    \[
    I = (10^{-3})^2 \times 10^5 \quad
    J = \frac{10^4 \times 10^5}{10^{11}} \quad
    k = \frac{0,1}{10^2} \quad
    P = 1000000 \quad
    M = (0,00001)^{-4}
    \]
    \[
    N = 0,001 \times \frac{1}{10^{-2}}
    \]
\end{enumerate}

\end{document}
