\documentclass[12pt]{article}
\usepackage{stmaryrd}
\usepackage{graphicx}
\usepackage[utf8]{inputenc}

\usepackage[french]{babel}
\usepackage[T1]{fontenc}
%\usepackage{hyperref}
\usepackage[colorlinks=true, linkcolor=blue, urlcolor=blue, citecolor=blue]{hyperref}
\usepackage{verbatim}

\usepackage{color, soul}

\usepackage{pgfplots}
\pgfplotsset{compat=1.15}
\usepackage{mathrsfs}

\usepackage{amsmath}
\usepackage{amsfonts}
\usepackage{amssymb}
\usepackage{tkz-tab}

\usepackage{tikz}
\usetikzlibrary{arrows, shapes.geometric, fit}


\usepackage[margin=2cm]{geometry}
\usepackage{eso-pic}         % Pour ajouter des éléments en arrière-plan

% Commande pour ajouter du texte en arrière-plan
\AddToShipoutPicture{
    \AtTextCenter{%
        \makebox[0pt]{\rotatebox{45}{\textcolor[gray]{0.9}{\fontsize{5cm}{5cm}\selectfont Pathé BA}}}
    }
}

\begin{document}

\begin{minipage}{0.8\textwidth}
	Talla                        
\end{minipage}
\begin{minipage}{0.8\textwidth}
	Diallo 
\end{minipage}

\begin{center}
\textbf{{\underline{\textcolor{green}{Correction}}}}
\end{center}
\section*{\textcolor{green}{\underline{Exercice 1}:}}
Traduit par une formule mathématique utilisant les quantificateurs les affirmations ci-dessous:

($f$ est une fonction numérique d'une variable réelle, $\ell$ un réel)
\begin{enumerate}
\item \[\lim_{x \to -\infty} f(x) = l\]

\item \[\lim_{x \to +\infty} f(x) = +\infty\]

\item \[\lim_{x \to -\infty} f(x) = +\infty\]

\item \[\lim_{x \to -2} \ln(2x+5) = 0\]

\item \[\lim_{x \to 2} \frac{x^{2}-3x+5}{(x-2)^{2}} = +\infty\]
\end{enumerate}
\section*{\textcolor{green}{\underline{Correction Exercice 1}:}}
1. 
\[
\forall \epsilon > 0, \exists M \in \mathbb{R}, \text{ tel que } \forall x < M, |f(x) - l| < \epsilon.
\]
Cette expression signifie que, pour tout $\epsilon$ positif, il existe un réel $M$ tel que pour tout $x$ inférieur à $M$, la différence entre $f(x)$ et ll est inférieure à$\epsilon$ , c'est-à-dire que $f(x)$ se rapproche de $\ell$ lorsque $x$ tend vers $-\infty$.

2.
\[
\forall A > 0, \exists M \in \mathbb{R}, \text{ tel que } \forall x > M, f(x) > A.
\]


3. 
\[
\forall A > 0, \exists M \in \mathbb{R}, \text{ tel que } \forall x < M, f(x) > A.
\]


4. 
\[
\forall \epsilon > 0, \exists \delta > 0, \text{ tel que } 0 < |x + 2| < \delta \implies |\ln(2x+5)| < \epsilon.
\]


5. 

\[
\forall A > 0, \exists \delta > 0, \text{ tel que } 0 < |x - 2| < \delta \implies \frac{x^{2}-3x+5}{(x-2)^{2}} > A.
\]

\section*{Exercice 2}

1. \text{En utilisant la définition, montrer que } 
\[\lim_{x \to 5} \sqrt{x} = \sqrt{5}\]

2. \text{Utiliser un encadrement pour déterminer la limite en } $+\infty$ \text{ de la fonction } f \text{ définie par }\\
\[f(x) = \frac{\mathrm{E}(x)}{x}\]
\section*{\textcolor{green}{\underline{Correction Exercice 2}:}}

1. \textbf{En utilisant la définition, montrer que}
\[
\lim_{x \to 5} \sqrt{x} = \sqrt{5}.
\]
\textbf{Solution} : Utilisons la définition formelle de la limite. Nous devons montrer que :
\[
\forall \epsilon > 0, \exists \delta > 0, \text{ tel que } 0 < |x - 5| < \delta \implies |\sqrt{x} - \sqrt{5}| < \epsilon.
\]
En d'autres termes, pour tout \( \epsilon > 0 \), il existe un \( \delta > 0 \) tel que si \( x \) est suffisamment proche de 5 (c'est-à-dire que \( |x - 5| < \delta \)), alors \( \sqrt{x} \) est proche de \( \sqrt{5} \) (c'est-à-dire que \( |\sqrt{x} - \sqrt{5}| < \epsilon \)).

Pour déterminer \( \delta \), on commence par manipuler l'inégalité \( |\sqrt{x} - \sqrt{5}| < \epsilon \) :

\[
|\sqrt{x} - \sqrt{5}| = \frac{|x - 5|}{\sqrt{x} + \sqrt{5}}.
\]

On cherche donc \( \delta \) tel que :
\[
\frac{|x - 5|}{\sqrt{x} + \sqrt{5}} < \epsilon.
\]

Puisque \( \sqrt{x} \) est proche de \( \sqrt{5} \) lorsque \( x \to 5 \), on peut majorer \( \sqrt{x} + \sqrt{5} \) par une constante, et ainsi choisir \( \delta \) en fonction de \( \epsilon \).

2. \textbf{Utiliser un encadrement pour déterminer la limite en } $+\infty$ \textbf{ de la fonction } f \textbf{ définie par } $f(x) = \frac{E(x)}{x}$.

\textbf{Solution} : Si \( E(x) \) désigne la partie entière de \( x \), on peut encadrer \( f(x) \) de la manière suivante :
\[
\frac{x-1}{x} \leq \frac{E(x)}{x} \leq \frac{x}{x} = 1.
\]

Ainsi, on a l'encadrement :
\[
1 - \frac{1}{x} \leq f(x) \leq 1.
\]

En passant à la limite lorsque \(x \to +\infty\), on obtient :
\[
\lim_{x \to +\infty} \left( 1 - \frac{1}{x} \right) = 1 \quad \text{et} \quad \lim_{x \to +\infty} 1 = 1.
\]

Par le théorème des gendarmes, on en déduit que :
\[
\lim_{x \to +\infty} f(x) = 1.
\]

\section*{Exercice 3}

1. \text{Quand dit-on qu'une fonction } \( f \) \text{ est minorée sur un intervalle } \( I \)\text{ ?}

2. \text{Quand dit-on qu'une fonction } \( f \)  est continue sur un intervalle

du type  \([a, b]\) \text{ avec } \(a, b \in \mathbb{R}\) \text{ ?}

3. \text{Donner la proposition exprimant la propriété appelée "Inégalité des accroissements finis".}

4. \text{Donner la proposition exprimant la propriété appelée "Règle de l'Hospital".}

5. \text{Justifier que } \[ \lim_{{x \to 0}} \frac{e^{x}-1}{\sin x} = 1 \]

6. \text{Montrer que si } \( f \) \text{ et } \( g \) \text{ sont deux fonctions croissantes, alors la fonction } \( f + g \)\\ \text{ est une fonction croissante.}

7.\text{Peut-on prolonger par continuité en } \( x_0 = 1\) \text{ la fonction définie par ?}
 \(f(x) = \frac{x^2 - 1}{x^3 - 1}\)
 
 \text{ Justifier.}

\end{document}