\documentclass[12pt]{article}
\usepackage{stmaryrd}
\usepackage{graphicx}
\usepackage[utf8]{inputenc}

\usepackage[french]{babel}
\usepackage[T1]{fontenc}
%\usepackage{hyperref}
\usepackage[colorlinks=true, linkcolor=blue, urlcolor=blue, citecolor=blue]{hyperref}
\usepackage{verbatim}

\usepackage{color, soul}

\usepackage{pgfplots}
\pgfplotsset{compat=1.15}
\usepackage{mathrsfs}

\usepackage{amsmath}
\usepackage{amsfonts}
\usepackage{amssymb}
\usepackage{tkz-tab}

\usepackage{tikz}
\usetikzlibrary{arrows, shapes.geometric, fit}


\usepackage[margin=2cm]{geometry}
\usepackage{eso-pic}         % Pour ajouter des éléments en arrière-plan

% Commande pour ajouter du texte en arrière-plan
\AddToShipoutPicture{
    \AtTextCenter{%
        \makebox[0pt]{\rotatebox{45}{\textcolor[gray]{0.9}{\fontsize{5cm}{5cm}\selectfont Pathé BA}}}
    }
}

\begin{document}

\begin{minipage}{0.8\textwidth}
	Talla                        
\end{minipage}
\begin{minipage}{0.8\textwidth}
	Diallo 
\end{minipage}

\begin{center}
\textbf{{\underline{\textcolor{green}{Correction}}}}
\end{center}
\section*{\textcolor{green}{\underline{Exercice 1}:}}
Traduit par une formule mathématique utilisant les quantificateurs les affirmations ci-dessous:

($f$ est une fonction numérique d'une variable réelle, $\ell$ un réel)
\begin{enumerate}
\item \[\lim_{x \to -\infty} f(x) = l\]

\item \[\lim_{x \to +\infty} f(x) = +\infty\]

\item \[\lim_{x \to -\infty} f(x) = +\infty\]

\item \[\lim_{x \to -2} \ln(2x+5) = 0\]

\item \[\lim_{x \to 2} \frac{x^{2}-3x+5}{(x-2)^{2}} = +\infty\]
\end{enumerate}
\section*{\textcolor{green}{\underline{Correction Exercice 1}:}}
1. 
\[
\forall \epsilon > 0, \exists M \in \mathbb{R}, \text{ tel que } \forall x < M, |f(x) - l| < \epsilon.
\]
Cette expression signifie que, pour tout $\epsilon$ positif, il existe un réel $M$ tel que pour tout $x$ inférieur à $M$, la différence entre $f(x)$ et ll est inférieure à$\epsilon$ , c'est-à-dire que $f(x)$ se rapproche de $\ell$ lorsque $x$ tend vers $-\infty$.

2.
\[
\forall A > 0, \exists M \in \mathbb{R}, \text{ tel que } \forall x > M, f(x) > A.
\]


3. 
\[
\forall A > 0, \exists M \in \mathbb{R}, \text{ tel que } \forall x < M, f(x) > A.
\]


4. 
\[
\forall \epsilon > 0, \exists \delta > 0, \text{ tel que } 0 < |x + 2| < \delta \implies |\ln(2x+5)| < \epsilon.
\]


5. 

\[
\forall A > 0, \exists \delta > 0, \text{ tel que } 0 < |x - 2| < \delta \implies \frac{x^{2}-3x+5}{(x-2)^{2}} > A.
\]

\section*{Exercice 2}

1. \text{En utilisant la définition, montrer que } 
\[\lim_{x \to 5} \sqrt{x} = \sqrt{5}\]

2. \text{Utiliser un encadrement pour déterminer la limite en } $+\infty$ \text{ de la fonction } f \text{ définie par }\\
\[f(x) = \frac{\mathrm{E}(x)}{x}\]

\end{document}