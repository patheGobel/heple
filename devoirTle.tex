\documentclass[12pt]{article}
\usepackage{stmaryrd}
\usepackage{graphicx}
\usepackage[utf8]{inputenc}

\usepackage[french]{babel}
\usepackage[T1]{fontenc}
\usepackage{hyperref}
\usepackage{verbatim}

\usepackage{color, soul}

\usepackage{pgfplots}
\pgfplotsset{compat=1.15}
\usepackage{mathrsfs}

\usepackage{amsmath}
\usepackage{amsfonts}
\usepackage{amssymb}
\usepackage{tkz-tab}

\usepackage{tikz}
\usetikzlibrary{arrows, shapes.geometric, fit}


\usepackage[margin=2cm]{geometry}
\begin{document}

\begin{minipage}{0.5\textwidth}
	Ministère de l'éducation nationale  \\
	Inspection académique de Kédougou   \\
	Classe : Tle  \\
\end{minipage}
\begin{minipage}{0.5\textwidth}
	Année scolaire 2023-2024 \\
	Date : 14-05-2024 \\
	Durée : 3h 00 \\
\end{minipage}

\begin{center}
	\textbf{{\underline{Devoir N2 Du Second Semestre}}}
\end{center}
\section*{\underline{Exercice 1: }\textbf{6 pts}}
\subsection*{ Resoudre dans $\mathbb{R}$ 1pt+1pt+1,5pts+1pt+1,5pts}
$\ln(2x-1)=\ln(3x+3)$

$\ln(x-1)+\ln(x+1)=\ln(x+2)$

$\ln(2x-1)+2\ln(x+1)=\ln(x-1)$

$\ln(x-1)<\ln(3-x)$

$\ln(1-x)-\ln(2x+3)\geq\ln(x-2)$
\section*{\underline{\textcolor{green}{Correction Exercice 1: \textbf{6 pts}}}}
\textbf{2. \(\ln(2x-1)=\ln(3x+3)\)}

\textbf{\underline{\textcolor{green}{Domaine de Validité: D}}}

L'équation n'a de sens que si $2x-1>0$ et $3x+3>0$

Posons $2x-1=0$ et $3x+3=0$

C'est-à-dire $x=\frac{1}{2}$ et $x=-1$

\definecolor{cqcqcq}{rgb}{0.7529411764705882,0.7529411764705882,0.7529411764705882}
\begin{tikzpicture}[line cap=round,line join=round,>=triangle 45,x=1cm,y=1cm]
%\draw [color=cqcqcq,, xstep=1cm,ystep=1cm] (-7,-10) grid (-22,17);
\clip(-22,3) rectangle (12,10);
\draw [line width=2pt] (-23,8)-- (-7,8); %première ligne A(-22,8)---B(-7,8)
\draw [line width=2pt] (-22,6)-- (-7,6); %deuxième ligne
\draw [line width=2pt] (-22,5)-- (-7,5); %troisième  ligne
\draw [line width=2pt] (-22,4)-- (-7,4); %quatrième ligne
\draw [line width=2pt] (-22,4)-- (-22,8); %première colonne (-22,4)<----(-22,8);
\draw [line width=2pt] (-18,8)-- (-18,4); %deuxième colone  (-18,8)--->(-18,4);
\draw [line width=2pt] (-7,8)-- (-7,4); %quatrième colonne (-7,8)-->(-7,4);
\draw (-21,7) node[anchor=north west] {$x$};
\draw (-18,7) node[anchor=north west] {$-\infty$};
\draw (-8,7) node[anchor=north west] {$+\infty$};
\draw (-21,5.7) node[anchor=north west] {$2x-1$};
\draw (-15.8,5.7) node[anchor=north west] {$-$};
\draw (-15.3,4.8) node[anchor=north west] {$O$};
\draw (-10.5,5.7) node[anchor=north west] {$+$};
\draw (-21,4.7) node[anchor=north west] {$3x+3$};
\draw (-15,5.7) node[anchor=north west] {$-$};
\draw (-15.8,4.7) node[anchor=north west] {$-$};
\draw (-11.3,5.8) node[anchor=north west] {$O$};
\draw (-10.5,4.7) node[anchor=north west] {$+$};
\draw (-15,4.7) node[anchor=north west] {$+$};
\draw [line width=2pt] (-15,6)-- (-15,4); %(-13,6)-- (-13,4);
\draw [line width=2pt] (-11,6)-- (-11,4); %(-13,6)-- (-13,4);
\draw (-15.5,7) node[anchor=north west] {$-1$};
\draw (-11.3,7) node[anchor=north west] {$\frac{1}{2}$};
\end{tikzpicture}

Donc D=$\left]\frac{1}{2} , +\infty\right[ $

\textbf{\underline{\textcolor{green}{Résolution}}}

$\ln(2x-1)=\ln(3x+3)\Longrightarrow 2x-1=3x+3\Longrightarrow x=-4$

Comme $-4\notin$D Donc \textcolor{green}{\boxed{S=\emptyset}}

--------------------------------------------------------------------------------------------------------------

\textbf{2. \(\ln(x-1)+\ln(x+1)=\ln(x+2)\)}

\textbf{\underline{\textcolor{green}{Domaine de Validité: D}}}

L'équation n'a de sens que si \(x-1>0\) et \(x+1>0\) et \(x+2>0\).

Posons \(x-1=0\) et \(x+1=0\) et \(x+2=0\).

C'est-à-dire \(x=1\) et \(x=-1\) et \(x=-2\)

\definecolor{cqcqcq}{rgb}{0.7529411764705882,0.7529411764705882,0.7529411764705882}
\begin{tikzpicture}[line cap=round,line join=round,>=triangle 45,x=1cm,y=1cm]
%\draw [color=cqcqcq,, xstep=1cm,ystep=1cm] (-7,-10) grid (-22,17);
\clip(-22,3) rectangle (12,10);
\draw [line width=2pt] (-23,8)-- (-7,8); %première ligne A(-22,8)---B(-7,8)
\draw [line width=2pt] (-22,6)-- (-7,6); %deuxième ligne
\draw [line width=2pt] (-22,5)-- (-7,5); %troisième  ligne
\draw [line width=2pt] (-22,4)-- (-7,4); %quatrième ligne
\draw [line width=2pt] (-22,3)-- (-7,3); %cinquième ligne
\draw [line width=2pt] (-22,3)-- (-22,8); %première colonne (-22,4)<----(-22,8);
\draw [line width=2pt] (-18,8)-- (-18,3); %deuxième colone  (-18,8)--->(-18,4);
\draw [line width=2pt] (-7,8)-- (-7,3); %quatrième colonne (-7,8)-->(-7,4);
\draw (-21,7) node[anchor=north west] {$x$};
\draw (-18,7) node[anchor=north west] {$-\infty$};
\draw (-8,7) node[anchor=north west] {$+\infty$};
\draw (-21,5.7) node[anchor=north west] {$x-1$};
\draw (-15.8,5.7) node[anchor=north west] {$-$};
\draw (-15,5.7) node[anchor=north west] {$-$};
\draw (-12,5.7) node[anchor=north west] {$-$};
\draw (-11.3,5.7) node[anchor=north west] {$O$};
\draw (-10.5,5.7) node[anchor=north west] {$+$};
\draw (-21,4.7) node[anchor=north west] {$x+1$};
\draw (-15.8,4.7) node[anchor=north west] {$-$};
\draw (-15,4.7) node[anchor=north west] {$-$};
\draw (-13.4,4.7) node[anchor=north west] {$O$};
\draw (-12,4.7) node[anchor=north west] {$+$};
\draw (-10.5,4.7) node[anchor=north west] {$+$};
\draw (-21,3.7) node[anchor=north west] {$x+2$};
\draw (-15.8,4) node[anchor=north west] {$-$};
\draw (-15.3,4) node[anchor=north west] {$O$};
\draw (-15,4) node[anchor=north west] {$+$};
\draw (-12,4) node[anchor=north west] {$+$};
\draw (-10.5,4) node[anchor=north west] {$+$};
\draw [line width=2pt] (-15,6)-- (-15,3); %(-13,6)-- (-13,4);
\draw [line width=2pt] (-11,6)-- (-11,3); %(-13,6)-- (-13,4);
\draw [line width=2pt] (-13.1,6)-- (-13.1,3); %(-13,6)-- (-13,4);
\draw (-15.5,7) node[anchor=north west] {$-2$};
\draw (-13.5,7) node[anchor=north west] {$-1$};
\draw (-11.3,7) node[anchor=north west] {$1$};
\end{tikzpicture}

Donc D=$]1;+\infty[$

\textbf{\underline{\textcolor{green}{Résolution}}}
\[
\ln(x-1)+\ln(x+1)=\ln(x+2)\Longrightarrow \ln((x-1)(x+1))=\ln(x+2)
\]
\[
(x-1)(x+1)=x+2 \Longrightarrow x^2-1=x+2 \Longrightarrow x^2-x-3=0
\]
\[
\Delta=13 \Longrightarrow x=\frac{1-\sqrt{13}}{2} \text{ ou } x=\frac{1+\sqrt{13}}{2}
\]

Comme $\frac{1+\sqrt{13}}{2}\in$D et $\frac{1-\sqrt{13}}{2}\notin$D Donc \textcolor{green}{\boxed{S=\left\lbrace \frac{1+\sqrt{13}}{2}\right\rbrace }}

-----------------------------------------------------------------------------------------------------------

\textbf{3. \(\ln(2x-1)+2\ln(x+1)=\ln(x-1)\)}

\textbf{\underline{\textcolor{green}{Domaine de Validité: D}}}

L'équation n'a de sens que si \(2x-1>0\), \(x+1>0\) et \(x-1>0\).

Posons \(2x-1=0\) et \(x+1=0\) et \(x-1=0\).

C'est-à-dire \(x=\frac{1}{2}\) et \(x=-1\) et \(x=1\)

\definecolor{cqcqcq}{rgb}{0.7529411764705882,0.7529411764705882,0.7529411764705882}
\begin{tikzpicture}[line cap=round,line join=round,>=triangle 45,x=1cm,y=1cm]
%\draw [color=cqcqcq,, xstep=1cm,ystep=1cm] (-7,-10) grid (-22,17);
\clip(-22,3) rectangle (12,10);
\draw [line width=2pt] (-23,8)-- (-7,8); %première ligne A(-22,8)---B(-7,8)
\draw [line width=2pt] (-22,6)-- (-7,6); %deuxième ligne
\draw [line width=2pt] (-22,5)-- (-7,5); %troisième  ligne
\draw [line width=2pt] (-22,4)-- (-7,4); %quatrième ligne
\draw [line width=2pt] (-22,3)-- (-7,3); %cinquième ligne
\draw [line width=2pt] (-22,3)-- (-22,8); %première colonne (-22,4)<----(-22,8);
\draw [line width=2pt] (-18,8)-- (-18,3); %deuxième colone  (-18,8)--->(-18,4);
\draw [line width=2pt] (-7,8)-- (-7,3); %quatrième colonne (-7,8)-->(-7,4);
\draw (-21,7) node[anchor=north west] {$x$};
\draw (-18,7) node[anchor=north west] {$-\infty$};
\draw (-8,7) node[anchor=north west] {$+\infty$};
\draw (-21,5.7) node[anchor=north west] {$x-1$};
\draw (-15.8,5.7) node[anchor=north west] {$-$};
\draw (-15,5.7) node[anchor=north west] {$-$};
\draw (-12,5.7) node[anchor=north west] {$-$};
\draw (-11.3,5.7) node[anchor=north west] {$O$};
\draw (-10.5,5.7) node[anchor=north west] {$+$};
\draw (-21,4.7) node[anchor=north west] {$2x-1$};
\draw (-15.8,4.7) node[anchor=north west] {$-$};
\draw (-15,4.7) node[anchor=north west] {$-$};
\draw (-13.4,4.7) node[anchor=north west] {$O$};
\draw (-12,4.7) node[anchor=north west] {$+$};
\draw (-10.5,4.7) node[anchor=north west] {$+$};
\draw (-21,3.7) node[anchor=north west] {$x+1$};
\draw (-15.8,4) node[anchor=north west] {$-$};
\draw (-15.3,4) node[anchor=north west] {$O$};
\draw (-15,4) node[anchor=north west] {$+$};
\draw (-12,4) node[anchor=north west] {$+$};
\draw (-10.5,4) node[anchor=north west] {$+$};
\draw [line width=2pt] (-15,6)-- (-15,3); %(-13,6)-- (-13,4);
\draw [line width=2pt] (-11,6)-- (-11,3); %(-13,6)-- (-13,4);
\draw [line width=2pt] (-13.1,6)-- (-13.1,3); %(-13,6)-- (-13,4);
\draw (-15.5,7) node[anchor=north west] {$-1$};
\draw (-13.5,7) node[anchor=north west] {$\frac{1}{2}$};
\draw (-11.3,7) node[anchor=north west] {$1$};
\end{tikzpicture}

Donc \(D = ]1, +\infty[\).

\textbf{\underline{\textcolor{green}{Résolution}}}

\[
\ln(2x-1)+2\ln(x+1)=\ln(x-1)\Longrightarrow \ln((2x-1)(x+1)^2)=\ln(x-1)
\]
\[
(2x-1)(x+1)^2=x-1
\]

\[
(2x-1)(x+1)^2=x-1 \Longrightarrow (2x-1)(x^2 + 2x + 1)=x-1
\]
\[
2x^3 + 4x^2 + 2x - x^2 - 2x - 1 = x - 1 \Longrightarrow 2x^3 + 3x^2 - 3x = 0
\]
\[
x(2x^2 + 3x - 3) = 0
\]
\[
x=0 \text{ ou } 2x^2 + 3x - 3 = 0
\]

nous résolvons \(2x^2 + 3x - 3 = 0\):

\[
\Delta = 3^2 - 4 \times 2 \times (-3) = 9 + 24 = 33
\]
\[
x = \frac{-3 \pm \sqrt{33}}{4}
\]

Nous obtenons \(x = \frac{-3 + \sqrt{33}}{4}\) et \(x = \frac{-3 - \sqrt{33}}{4}\).

Comme \(0 \notin D\) ; \(\frac{-3 + \sqrt{33}}{4} \in D\) et \(\frac{-3 - \sqrt{33}}{4} \notin D\)  

Donc \(S = \left\{\frac{-3 + \sqrt{33}}{4}\right\}\).

\textcolor{green}{\boxed{S= \left\{\frac{-3 + \sqrt{33}}{4}\right\}}}

-----------------------------------------------------------------------------------------------------

\textbf{4. \(\ln(x-1)<\ln(3-x)\)}

\textbf{\underline{\textcolor{green}{Domaine de Validité: D}}}

L'équation n'a de sens que si \(x-1>0\) et \(3-x>0\).

C'est-à-dire \(1<x<3\).

\textbf{\underline{\textcolor{green}{Résolution}}}

\[
\ln(x-1)<\ln(3-x)\Longrightarrow x-1<3-x
\]
\[
2x<4\Longrightarrow x<2
\]

\textcolor{green}{\boxed{S=]1,2[}}

------------------------------------------------------------------------------------------------------------

\textbf{5. \(\ln(1-x)-\ln(2x+3)\geq\ln(x-2)\)}

\textbf{\underline{\textcolor{green}{Domaine de Validité: D}}}

L'équation n'a de sens que si \(1-x>0\), \(2x+3>0\) et \(x-2>0\).

C'est-à-dire \(x<1\), \(x>-\frac{3}{2}\) et \(x>2\).

Comme \(x\) ne peut pas satisfaire simultanément ces trois conditions, nous avons :

\[
D = \emptyset \Longrightarrow S = \emptyset
\]

\section*{\underline{Exercice 2: }\textbf{6 pts}}
\subsection*{ 1) Développer } $(x-1)(2x+1)(x+3)$
\subsection*{ 2) Résoudre } $2e^{3x}+5e^{2x}-4e^{x}-3$
\subsection*{ 3) Résoudre } $8x^{4}-6x^{2}+1=0$ puis $8e^{4x}-6e^{2x}+1=0$
\subsection*{ 4) Développer } $(3-x)(2x+1)$ et $(x-2)(3-x)(2x+1)$
\subsection*{ 5) Développer } $3e^{-2x}-5e^{-x}-2=0$ ; $-2e^{3x+1}+9e^{2x+1}+-7e^{x+1}-6e=0$
\subsection*{ 5) Développer $\mathbb{R}^{2}$} 
\( \begin{cases}
x + y = 3 \\
\ln x + \ln y = \ln 2
\end{cases}\)
\section*{\underline{\textcolor{green}{Correction Exercice 2: \textbf{6 pts}}}}

\section*{\underline{Exercice 2: }\textbf{6 pts}}

\subsection*{1) Développer } 
\[
(x-1)(2x+1)(x+3)
\]

\subsection*{Solution}
Développons pas à pas :
\[
(x-1)(2x+1)(x+3)
\]
\[
(x-1)(2x^2 + 7x + 3)
\]
\[
x(2x^2 + 7x + 3) - 1(2x^2 + 7x + 3)
\]
\[
2x^3 + 7x^2 + 3x - 2x^2 - 7x - 3
\]
\[
2x^3 + 5x^2 - 4x - 3
\]

\subsection*{2) Résoudre } 
\[
2e^{3x}+5e^{2x}-4e^{x}-3 = 0
\]

\subsection*{Solution}
Notons \( u = e^x \). L'équation devient alors :
\[
2u^3 + 5u^2 - 4u - 3 = 0
\]
Cette équation cubique est la forme dévelloppée de $(x-1)(2x+1)(x+3)$.

Donc $2u^3 + 5u^2 - 4u - 3=(u-1)(2u+1)(u+3)=(e^x-1)(2e^x+1)(e^x+3)$
\[
2u^3 + 5u^2 - 4u - 3 = 0 \Longrightarrow e^x=1 \text{ ou } e^x=-\frac{1}{2} \text{ ou } e^x=-3
\]
D'où \textcolor{green}{\boxed{S= \left\{0\right\}}}
\subsection*{3) Résoudre} 
\[
8x^4 - 6x^2 + 1 = 0
\]
puis 
\[
8e^{4x} - 6e^{2x} + 1 = 0
\]

\subsection*{Solution}
Pour \(8x^4 - 6x^2 + 1 = 0\) :
\[
\text{Posons } y = x^2, \text{ ainsi } 8y^2 - 6y + 1 = 0
\]
Résolvons cette équation quadratique pour \(y\).

L'équation quadratique est :
\[
8y^2 - 6y + 1 = 0
\]

Les solutions de cette équation sont données par la formule quadratique :
\[
y = \frac{-b \pm \sqrt{b^2 - 4ac}}{2a}
\]

Ici, \(a = 8\), \(b = -6\), et \(c = 1\). Calculons le discriminant :
\[
\Delta = b^2 - 4ac = (-6)^2 - 4 \cdot 8 \cdot 1 = 36 - 32 = 4
\]

Puis, trouvons les racines :
\[
y = \frac{6 \pm \sqrt{4}}{2 \cdot 8} = \frac{6 \pm 2}{16}
\]

Cela donne les solutions :
\[
y_1 = \frac{6 + 2}{16} = \frac{8}{16} = \frac{1}{2}
\]
\[
y_2 = \frac{6 - 2}{16} = \frac{4}{16} = \frac{1}{4}
\]

Comme \(y = x^2\), nous avons :
\[
x^2 = \frac{1}{2} \implies x = \pm \sqrt{\frac{1}{2}} = \pm \frac{\sqrt{2}}{2}
\]
\[
x^2 = \frac{1}{4} \implies x = \pm \frac{1}{2}
\]

Ainsi, les solutions pour \(x\) sont :
\[
x = \pm \frac{\sqrt{2}}{2}, \quad x = \pm \frac{1}{2}
\]

\textcolor{green}{\boxed{S=\left\lbrace -\frac{\sqrt{2}}{2},\frac{\sqrt{2}}{2}, -\frac{1}{2}, \frac{1}{2}\right\rbrace }}

La forme dévélopée de \(8x^4 - 6x^2 + 1\) est \((x-\frac{\sqrt{2}}{2})(x+\frac{\sqrt{2}}{2})(x-\frac{1}{2})(x+\frac{1}{2})\)

Pour \(8e^{4x} - 6e^{2x} + 1 = 0\) :
\[
\text{Posons } u = e^{2x}, \text{ ainsi } 8u^2 - 6u + 1 = 0
\]
Cette équation cubique est la forme dévelloppée de \((u-\frac{\sqrt{2}}{2})(u+\frac{\sqrt{2}}{2})(u-\frac{1}{2})(u+\frac{1}{2})\).

Donc $8u^2 - 6u + 1 - 3=(u-\frac{\sqrt{2}}{2})(u+\frac{\sqrt{2}}{2})(u-\frac{1}{2})(u+\frac{1}{2})$
\[
8u^2 - 6u + 1 - 3 = 0 \Longrightarrow e^{2x}=\frac{\sqrt{2}}{2} \text{ ou } e^{2x}=-\frac{\sqrt{2}}{2} \text{impossible ou } e^{2x}=\frac{1}{2} \text{ ou } e^{2x}=-\frac{1}{2} \text{impossible}
\]

\[
2x=\ln(\frac{\sqrt{2}}{2}) \text{ ou } 2x=\ln(\frac{1}{2})
\]

\[
x=\frac{\ln(\frac{\sqrt{2}}{2})}{2} \text{ ou } x=\frac{\ln(\frac{1}{2})}{2}
\]
\[
x=\frac{\ln(\frac{\sqrt{2}}{2})}{2} \text{ ou } x=\frac{\ln(\frac{1}{2})}{2}
\]
D'où \textcolor{green}{\boxed{S= \left\{\frac{\ln(\frac{\sqrt{2}}{2})}{2}, \frac{\ln(\frac{1}{2})}{2}\right\}}}

\subsection*{4) Développer } 
\[
(3-x)(2x+1) \text{ et } (x-2)(3-x)(2x+1)
\]

\subsection*{Solution}
\textcolor{green}{\underline{Pour \((3-x)(2x+1)\)} }:
\[
3(2x+1) - x(2x+1)
\]
\[
6x + 3 - 2x^2 - x
\]
\[
-2x^2 + 5x + 3
\]
Donc, l'expression développée est :
\[\textcolor{green}{\boxed{
(3-x)(2x+1) = -2x^2 + 5x + 3
}}\]
\textcolor{green}{\underline{Pour \((x-2)(3-x)(2x+1)\)} }:
\[
(x-2)(-2x^2 + 5x + 3)
\]
Développons cette expression pas à pas.

1. Écrivons l'expression initiale :
\[
(x-2)(-2x^2 + 5x + 3)
\]

2. Utilisons la distributivité pour développer chaque terme de \((x-2)\) avec chaque terme de \((-2x^2 + 5x + 3)\) :
\[
x(-2x^2 + 5x + 3) - 2(-2x^2 + 5x + 3)
\]

3. Distribuons \(x\) dans \((-2x^2 + 5x + 3)\) :
\[
x \cdot (-2x^2) + x \cdot 5x + x \cdot 3 = -2x^3 + 5x^2 + 3x
\]

4. Distribuons \(-2\) dans \((-2x^2 + 5x + 3)\) :
\[
-2 \cdot (-2x^2) + (-2) \cdot 5x + (-2) \cdot 3 = 4x^2 - 10x - 6
\]

5. Combinons les termes développés :
\[
-2x^3 + 5x^2 + 3x + 4x^2 - 10x - 6
\]

6. Simplifions en combinant les termes similaires :
\[
-2x^3 + (5x^2 + 4x^2) + (3x - 10x) - 6
\]
\[
-2x^3 + 9x^2 - 7x - 6
\]

Donc, l'expression développée est :
\[\textcolor{green}{\boxed{
(x-2)(-2x^2 + 5x + 3) = -2x^3 + 9x^2 - 7x - 6
}}\]

\subsection*{5) Résolvons } 
\[
3e^{-2x}-5e^{-x}-2=0 \quad ; \quad -2e^{3x+1}+9e^{2x+1}-7e^{x+1}-6e=0
\]

\subsection*{Solution}
\textcolor{green}{\underline{Pour \(3e^{-2x}-5e^{-x}-2=0\)}}:
\[
\text{Posons } u = e^{-x}, \text{ ainsi } 3u^2 - 5u - 2 = 0
\]

Cette équation cubique est la forme dévelloppée de $(3-u)(2u+1)$.

Donc $3e^{-2x}-5e^{-x}-2=(3-e^{-x})(2e^{-x}+1)$
\[
(3-e^{-x})(2e^{-x}+1)= 0 \Longrightarrow e^{-x}=3 \text{ ou } e^{-x}=-\frac{1}{2} \text{ impossible }
\]
\[
x=-\ln(3)
\]
D'où \textcolor{green}{\boxed{S= \left\{-\ln(3)\right\}}}

\textcolor{green}{\underline{Pour \(-2e^{3x+1}+9e^{2x+1}-7e^{x+1}-6e=0\) }}:
\[
\text{Posons } v = e^{x}, \text{ ainsi } -2v^3e + 9v^2e - 7ve - 6e = 0 \Longrightarrow -2v^3 + 9v^2 - 7v - 6 = 0
\]
Cette équation cubique est la forme dévelloppée de $(u-2)(3-u)(2u+1)$.

Donc $-2e^{3x+1}+9e^{2x+1}-7e^{x+1}-6e=(e^{x}-2)(3-e^{x})(2e^{x}+1)$
\[
(e^{x}-2)(3-e^{x})(2e^{x}+1)= 0 \Longrightarrow e^{x}=2 \text{ ou } e^{x}=3 \text{ ou } e^{x}=-\frac{1}{2} \text{ impossible }
\]
\[
x=\ln(2) \text{ ou }x=\ln(3), 
\]
D'où \textcolor{green}{\boxed{S= \left\{\ln(2),\ln(3)\right\}}}

\subsection*{6) Résoudre $\mathbb{R}^{2}$ } 
\[
\begin{cases}
x + y = 3 \\
\ln x + \ln y = \ln 2
\end{cases}
\]

\subsection*{Solution}
La première équation donne \( y = 3 - x \). En substituant cela dans la deuxième équation :
\[
\ln x + \ln (3 - x) = \ln 2
\]
\[
\ln (x(3 - x)) = \ln 2
\]
\[
x(3 - x) = 2
\]
\[
3x - x^2 = 2
\]
\[
x^2 - 3x + 2 = 0
\]
Résolvons cette équation pour \(x\), puis trouvons \(y\).



\section*{\underline{Problème: }\textbf{8 pts}}
Soit $f(x)=\ln(x^{2}+4x+4)$

1) a-Montrer que l'esemble de définition de $f$ est $Df=\mathbb{R}\setminus\left\lbrace -2 \right\rbrace $ et détermine les limites aux bornes de $Df$.$\textbf{0,5pt+1pt}$

b-Etuider les variations de $f$.$\textbf{1,5pt}$

2)Soit la courbe (Cf) représentative de $f$ dans un repère orthonormé (unité 1 cm).

a- Déterminer les points d'intersections de $Cf$ avec les axes du repère.$\textbf{1pt}$

b-Ecrire une équation de la tangente (T) à (Cf) au point d'abscisse 0.$\textbf{0,5pt}$

c- Montrer que la droite d'équation $x=-2$ est axe de  symétrie de (Cf). $\textbf{1pt}$

d-Tracer (Cf) et la tangente (T). $\textbf{1,5pt}$

3) Montrer que $f(x)=2\ln(x+2)$ sur $ \left]-2 +\infty \right[ $. $\textbf{1pt}$
\end{document}
