\documentclass[12pt]{article}
\usepackage{stmaryrd}
\usepackage{graphicx}
\usepackage[utf8]{inputenc}

\usepackage[french]{babel}
\usepackage[T1]{fontenc}
\usepackage{hyperref}
\usepackage{verbatim}

\usepackage{color, soul}

\usepackage{pgfplots}
\pgfplotsset{compat=1.15}
\usepackage{mathrsfs}

\usepackage{amsmath}
\usepackage{amsfonts}
\usepackage{amssymb}
\usepackage{tkz-tab}

\usepackage{tikz}
\usetikzlibrary{arrows, shapes.geometric, fit}


\usepackage[margin=2cm]{geometry}
\begin{document}

\begin{minipage}{0.5\textwidth}
	Ministère de l'éducation nationale  \\
	Inspection académique de Kédougou   \\
	Classe : Tle  \\
\end{minipage}
\begin{minipage}{0.5\textwidth}
	Année scolaire 2023-2024 \\
	Date : 14-05-2024 \\
	Durée : 3h 00 \\
\end{minipage}

\begin{center}
	\textbf{{\underline{Devoir N2 Du Second Semestre}}}
\end{center}
\section*{\underline{Exercice 1: }\textbf{6 pts}}
\subsection*{ Resoudre dans $\mathbb{R}$ 1pt+1pt+1,5pts+1pt+1,5pts}
$\ln(2x-1)=\ln(3x+3)$

$\ln(x-1)+\ln(x+1)=\ln(x+2)$

$\ln(2x-1)+2\ln(x+1)=\ln(x-1)$

$\ln(x-1)<\ln(3-x)$

$\ln(1-x)-\ln(2x+3)\geq\ln(x-2)$
\section*{\underline{\textcolor{green}{Correction Exercice 1: \textbf{6 pts}}}}
$\ln(2x-1)=\ln(3x+3)$

L'équation n'a de sens que si $2x-1>0$ et $3x+3>0$

Posons $2x-1=0$ et $3x+3=0$

C'est-à-dire $x=\frac{1}{2}$ et $x=-1$

\definecolor{cqcqcq}{rgb}{0.7529411764705882,0.7529411764705882,0.7529411764705882}
\begin{tikzpicture}[line cap=round,line join=round,>=triangle 45,x=1cm,y=1cm]
\draw [color=cqcqcq,, xstep=1cm,ystep=1cm] (-7,-10) grid (-22,17);
\clip(-22,3) rectangle (12,10);
\draw [line width=2pt] (-23,8)-- (-7,8); %première ligne A(-22,8)---B(-7,8)
\draw [line width=2pt] (-22,6)-- (-7,6); %deuxième ligne
\draw [line width=2pt] (-22,5)-- (-7,5); %troisième  ligne
\draw [line width=2pt] (-22,4)-- (-7,4); %quatrième ligne
\draw [line width=2pt] (-22,4)-- (-22,8); %première colonne (-22,4)<----(-22,8);
\draw [line width=2pt] (-18,8)-- (-18,4); %deuxième colone  (-18,8)--->(-18,4);
\draw [line width=2pt] (-7,8)-- (-7,4); %troisième colonne (-7,8)-->(-7,4);
\draw (-21,5.7) node[anchor=north west] {$2x-1$};
\draw (-21,7) node[anchor=north west] {$x$};
\draw (-18,7) node[anchor=north west] {$-\infty$};
\draw (-8,7) node[anchor=north west] {$+\infty$};
\draw (-15.8,5.3) node[anchor=north west] {$+$};
\draw (-10.5,5.3) node[anchor=north west] {$-$};
\draw [line width=2pt] (-13,6)-- (-13,4); %(-13,6)-- (-13,4);
\draw (-13.2,7) node[anchor=north west] {$\frac{29}{10}$};
\end{tikzpicture}

\section*{\underline{Exercice 2: }\textbf{6 pts}}
\subsection*{ 1) Développer } $(x-1)(2x+1)(x+3)$
\subsection*{ 2) Résoudre } $2e^{3x}+5e^{2x}-4e^{x}-3$
\subsection*{ 3) Développer } $8x^{4}-6x^{2}+1=0$ puis $8e^{4x}-6e^{2x}+1=0$
\subsection*{ 4) Développer } $(3-x)(2x+1)$ et $(x-2)(3-x)(2x+1)$
\subsection*{ 5) Développer } $3e^{-2x}-5e^{-x}-2=0$ ; $-2e^{3x+1}+9e^{2x+1}+-7e^{x+1}-6e=0$
\subsection*{ 5) Développer $\mathbb{R}^{2}$} 
\( \begin{cases}
x + y = 3 \\
\ln x + \ln y = \ln 2
\end{cases}\)
\section*{\underline{\textcolor{green}{Correction Exercice 2: \textbf{6 pts}}}}
\section*{\underline{Problème: }\textbf{8 pts}}
Soit $f(x)=\ln(x^{2}+4x+4)$

1) a-Montrer que l'esemble de définition de $f$ est $Df=\mathbb{R}\setminus\left\lbrace -2 \right\rbrace $ et détermine les limites aux bornes de $Df$.$\textbf{0,5pt+1pt}$

b-Etuider les variations de $f$.$\textbf{1,5pt}$

2)Soit la courbe (Cf) représentative de $f$ dans un repère orthonormé (unité 1 cm).

a- Déterminer les points d'intersections de $Cf$ avec les axes du repère.$\textbf{1pt}$

b-Ecrire une équation de la tangente (T) à (Cf) au point d'abscisse 0.$\textbf{0,5pt}$

c- Montrer que la droite d'équation $x=-2$ est axe de  symétrie de (Cf). $\textbf{1pt}$

d-Tracer (Cf) et la tangente (T). $\textbf{1,5pt}$

3) Montrer que $f(x)=2\ln(x+2)$ sur $ \left]-2 +\infty \right[ $. $\textbf{1pt}$
\end{document}