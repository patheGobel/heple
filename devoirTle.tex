\documentclass[12pt]{article}
\usepackage{stmaryrd}
\usepackage{graphicx}
\usepackage[utf8]{inputenc}

\usepackage[french]{babel}
\usepackage[T1]{fontenc}
\usepackage{hyperref}
\usepackage{verbatim}

\usepackage{color, soul}

\usepackage{pgfplots}
\pgfplotsset{compat=1.15}
\usepackage{mathrsfs}

\usepackage{amsmath}
\usepackage{amsfonts}
\usepackage{amssymb}
\usepackage{tkz-tab}

\usepackage{tikz}
\usetikzlibrary{arrows, shapes.geometric, fit}


\usepackage[margin=2cm]{geometry}
\begin{document}

\begin{minipage}{0.5\textwidth}
	Ministère de l'éducation nationale  \\
	Inspection académique de Kédougou   \\
	Classe : Tle  \\
\end{minipage}
\begin{minipage}{0.5\textwidth}
	Année scolaire 2023-2024 \\
	Date : 14-05-2024 \\
	Durée : 3h 00 \\
\end{minipage}

\begin{center}
	\textbf{{\underline{Devoir N2 Du Second Semestre}}}
\end{center}
\section*{\underline{Exercice 1: }\textbf{6 pts}}
\subsection*{ Resoudre dans $\mathbb{R}$ 1pt+1pt+1,5pts+1pt+1,5pts}
$\ln(2x-1)=\ln(3x+3)$

$\ln(x-1)+\ln(x+1)=\ln(x+2)$

$\ln(2x-1)+2\ln(x+1)=\ln(x-1)$

$\ln(x-1)<\ln(3-x)$

$\ln(1-x)-\ln(2x+3)\geq\ln(x-2)$
\section*{\underline{\textcolor{green}{Correction Exercice 1: \textbf{6 pts}}}}
$\ln(2x-1)=\ln(3x+3)$

\textbf{\underline{\textcolor{green}{Domaine de Validité: D}}}

L'équation n'a de sens que si $2x-1>0$ et $3x+3>0$

Posons $2x-1=0$ et $3x+3=0$

C'est-à-dire $x=\frac{1}{2}$ et $x=-1$

\definecolor{cqcqcq}{rgb}{0.7529411764705882,0.7529411764705882,0.7529411764705882}
\begin{tikzpicture}[line cap=round,line join=round,>=triangle 45,x=1cm,y=1cm]
%\draw [color=cqcqcq,, xstep=1cm,ystep=1cm] (-7,-10) grid (-22,17);
\clip(-22,3) rectangle (12,10);
\draw [line width=2pt] (-23,8)-- (-7,8); %première ligne A(-22,8)---B(-7,8)
\draw [line width=2pt] (-22,6)-- (-7,6); %deuxième ligne
\draw [line width=2pt] (-22,5)-- (-7,5); %troisième  ligne
\draw [line width=2pt] (-22,4)-- (-7,4); %quatrième ligne
\draw [line width=2pt] (-22,4)-- (-22,8); %première colonne (-22,4)<----(-22,8);
\draw [line width=2pt] (-18,8)-- (-18,4); %deuxième colone  (-18,8)--->(-18,4);
\draw [line width=2pt] (-7,8)-- (-7,4); %quatrième colonne (-7,8)-->(-7,4);
\draw (-21,7) node[anchor=north west] {$x$};
\draw (-18,7) node[anchor=north west] {$-\infty$};
\draw (-8,7) node[anchor=north west] {$+\infty$};
\draw (-21,5.7) node[anchor=north west] {$2x-1$};
\draw (-15.8,5.7) node[anchor=north west] {$-$};
\draw (-15.3,4.8) node[anchor=north west] {$O$};
\draw (-10.5,5.7) node[anchor=north west] {$+$};
\draw (-21,4.7) node[anchor=north west] {$3x+3$};
\draw (-15,5.7) node[anchor=north west] {$-$};
\draw (-15.8,4.7) node[anchor=north west] {$-$};
\draw (-11.3,5.8) node[anchor=north west] {$O$};
\draw (-10.5,4.7) node[anchor=north west] {$+$};
\draw (-15,4.7) node[anchor=north west] {$+$};
\draw [line width=2pt] (-15,6)-- (-15,4); %(-13,6)-- (-13,4);
\draw [line width=2pt] (-11,6)-- (-11,4); %(-13,6)-- (-13,4);
\draw (-15.5,7) node[anchor=north west] {$-1$};
\draw (-11.3,7) node[anchor=north west] {$\frac{1}{2}$};
\end{tikzpicture}

Donc D=$\left]\frac{1}{2} , +\infty\right[ $

\textbf{\underline{\textcolor{green}{Résolution}}}

$\ln(2x-1)=\ln(3x+3)\Longrightarrow 2x-1=3x+3\Longrightarrow x=-4$

Comme $x\notin$D Donc S=$\emptyset$

\textbf{2. \(\ln(x-1)+\ln(x+1)=\ln(x+2)\)}

\textbf{\underline{\textcolor{green}{Domaine de Validité: D}}}

L'équation n'a de sens que si \(x-1>0\) et \(x+1>0\) et \(x+2>0\).

Posons \(x-1=0\) et \(x+1=0\) et \(x+2=0\).

C'est-à-dire \(x=1\) et \(x=-1\) et \(x=-2\)

\definecolor{cqcqcq}{rgb}{0.7529411764705882,0.7529411764705882,0.7529411764705882}
\begin{tikzpicture}[line cap=round,line join=round,>=triangle 45,x=1cm,y=1cm]
%\draw [color=cqcqcq,, xstep=1cm,ystep=1cm] (-7,-10) grid (-22,17);
\clip(-22,3) rectangle (12,10);
\draw [line width=2pt] (-23,8)-- (-7,8); %première ligne A(-22,8)---B(-7,8)
\draw [line width=2pt] (-22,6)-- (-7,6); %deuxième ligne
\draw [line width=2pt] (-22,5)-- (-7,5); %troisième  ligne
\draw [line width=2pt] (-22,4)-- (-7,4); %quatrième ligne
\draw [line width=2pt] (-22,3)-- (-7,3); %cinquième ligne
\draw [line width=2pt] (-22,3)-- (-22,8); %première colonne (-22,4)<----(-22,8);
\draw [line width=2pt] (-18,8)-- (-18,3); %deuxième colone  (-18,8)--->(-18,4);
\draw [line width=2pt] (-7,8)-- (-7,3); %quatrième colonne (-7,8)-->(-7,4);
\draw (-21,7) node[anchor=north west] {$x$};
\draw (-18,7) node[anchor=north west] {$-\infty$};
\draw (-8,7) node[anchor=north west] {$+\infty$};
\draw (-21,5.7) node[anchor=north west] {$x-1$};
\draw (-15.8,5.7) node[anchor=north west] {$-$};
\draw (-15.3,4.8) node[anchor=north west] {$O$};
\draw (-10.5,5.7) node[anchor=north west] {$+$};
\draw (-21,4.7) node[anchor=north west] {$x+1$};
\draw (-15,5.7) node[anchor=north west] {$-$};
\draw (-15.8,4.7) node[anchor=north west] {$-$};
\draw (-11.3,5.8) node[anchor=north west] {$O$};
\draw (-10.5,4.7) node[anchor=north west] {$+$};
\draw (-15,4.7) node[anchor=north west] {$+$};
\draw (-21,3.7) node[anchor=north west] {$x+2$};
\draw [line width=2pt] (-15,6)-- (-15,3); %(-13,6)-- (-13,4);
\draw [line width=2pt] (-11,6)-- (-11,3); %(-13,6)-- (-13,4);
\draw [line width=2pt] (-13.1,6)-- (-13.1,3); %(-13,6)-- (-13,4);
\draw (-15.5,7) node[anchor=north west] {$-2$};
\draw (-13.5,7) node[anchor=north west] {$-1$};
\draw (-11.3,7) node[anchor=north west] {$1$};
\end{tikzpicture}

Donc D=$kk$
\textbf{\underline{\textcolor{green}{Résolution}}}
\[
\ln(x-1)+\ln(x+1)=\ln(x+2)\Longrightarrow \ln((x-1)(x+1))=\ln(x+2)
\]
\[
(x-1)(x+1)=x+2 \Longrightarrow x^2-1=x+2 \Longrightarrow x^2-x-3=0
\]
\[
(x-3)(x+1)=0 \Longrightarrow x=3 \text{ ou } x=-1
\]

Comme \(x \notin D\) pour \(x=-1\), Donc \(S=\{3\}\).

\textbf{3. \(\ln(2x-1)+2\ln(x+1)=\ln(x-1)\)}

\textbf{\underline{\textcolor{green}{Domaine de Validité: D}}}

L'équation n'a de sens que si \(2x-1>0\), \(x+1>0\) et \(x-1>0\).

C'est-à-dire \(x>\frac{1}{2}\) et \(x>1\).

Donc \(D = ]1, +\infty[\).

\textbf{\underline{\textcolor{green}{Résolution}}}

\[
\ln(2x-1)+2\ln(x+1)=\ln(x-1)\Longrightarrow \ln((2x-1)(x+1)^2)=\ln(x-1)
\]
\[
(2x-1)(x+1)^2=x-1
\]

Comme \(x > 1\), on peut simplifier :

\[
(2x-1)(x+1)^2=x-1 \Longrightarrow (2x-1)(x^2 + 2x + 1)=x-1
\]
\[
2x^3 + 5x^2 + x - 2x^2 - 2x - 1 = x - 1 \Longrightarrow 2x^3 + 3x^2 - 3x = 0
\]
\[
x(2x^2 + 3x - 3) = 0
\]
\[
x=0 \text{ ou } 2x^2 + 3x - 3 = 0
\]

Comme \(x \notin D\) pour \(x=0\), nous résolvons \(2x^2 + 3x - 3 = 0\):

\[
\Delta = 3^2 - 4 \times 2 \times (-3) = 9 + 24 = 33
\]
\[
x = \frac{-3 \pm \sqrt{33}}{4}
\]

Nous obtenons \(x = \frac{-3 + \sqrt{33}}{4}\) et \(x = \frac{-3 - \sqrt{33}}{4}\).

Comme \(x \notin D\) pour \(x = \frac{-3 - \sqrt{33}}{4}\), Donc \(S = \left\{\frac{-3 + \sqrt{33}}{4}\right\}\).

\textbf{4. \(\ln(x-1)<\ln(3-x)\)}

\textbf{\underline{\textcolor{green}{Domaine de Validité: D}}}

L'équation n'a de sens que si \(x-1>0\) et \(3-x>0\).

C'est-à-dire \(1<x<3\).

\textbf{\underline{\textcolor{green}{Résolution}}}

\[
\ln(x-1)<\ln(3-x)\Longrightarrow x-1<3-x
\]
\[
2x<4\Longrightarrow x<2
\]

Donc \(S=]1,2[\).

\textbf{5. \(\ln(1-x)-\ln(2x+3)\geq\ln(x-2)\)}

\textbf{\underline{\textcolor{green}{Domaine de Validité: D}}}

L'équation n'a de sens que si \(1-x>0\), \(2x+3>0\) et \(x-2>0\).

C'est-à-dire \(x<1\), \(x>-\frac{3}{2}\) et \(x>2\).

Comme \(x\) ne peut pas satisfaire simultanément ces trois conditions, nous avons :

\[
D = \emptyset \Longrightarrow S = \emptyset
\]

\section*{\underline{Exercice 2: }\textbf{6 pts}}
\subsection*{ 1) Développer } $(x-1)(2x+1)(x+3)$
\subsection*{ 2) Résoudre } $2e^{3x}+5e^{2x}-4e^{x}-3$
\subsection*{ 3) Développer } $8x^{4}-6x^{2}+1=0$ puis $8e^{4x}-6e^{2x}+1=0$
\subsection*{ 4) Développer } $(3-x)(2x+1)$ et $(x-2)(3-x)(2x+1)$
\subsection*{ 5) Développer } $3e^{-2x}-5e^{-x}-2=0$ ; $-2e^{3x+1}+9e^{2x+1}+-7e^{x+1}-6e=0$
\subsection*{ 5) Développer $\mathbb{R}^{2}$} 
\( \begin{cases}
x + y = 3 \\
\ln x + \ln y = \ln 2
\end{cases}\)
\section*{\underline{\textcolor{green}{Correction Exercice 2: \textbf{6 pts}}}}
\section*{\underline{Problème: }\textbf{8 pts}}
Soit $f(x)=\ln(x^{2}+4x+4)$

1) a-Montrer que l'esemble de définition de $f$ est $Df=\mathbb{R}\setminus\left\lbrace -2 \right\rbrace $ et détermine les limites aux bornes de $Df$.$\textbf{0,5pt+1pt}$

b-Etuider les variations de $f$.$\textbf{1,5pt}$

2)Soit la courbe (Cf) représentative de $f$ dans un repère orthonormé (unité 1 cm).

a- Déterminer les points d'intersections de $Cf$ avec les axes du repère.$\textbf{1pt}$

b-Ecrire une équation de la tangente (T) à (Cf) au point d'abscisse 0.$\textbf{0,5pt}$

c- Montrer que la droite d'équation $x=-2$ est axe de  symétrie de (Cf). $\textbf{1pt}$

d-Tracer (Cf) et la tangente (T). $\textbf{1,5pt}$

3) Montrer que $f(x)=2\ln(x+2)$ sur $ \left]-2 +\infty \right[ $. $\textbf{1pt}$
\end{document}
