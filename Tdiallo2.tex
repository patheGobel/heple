\documentclass[12pt]{article}
\usepackage{stmaryrd}
\usepackage{graphicx}
\usepackage[utf8]{inputenc}

\usepackage[french]{babel}
\usepackage[T1]{fontenc}
%\usepackage{hyperref}
\usepackage[colorlinks=true, linkcolor=blue, urlcolor=blue, citecolor=blue]{hyperref}
\usepackage{verbatim}

\usepackage{color, soul}

\usepackage{pgfplots}
\pgfplotsset{compat=1.15}
\usepackage{mathrsfs}

\usepackage{amsmath}
\usepackage{amsfonts}
\usepackage{amssymb}
\usepackage{tkz-tab}

\usepackage{tikz}
\usetikzlibrary{arrows, shapes.geometric, fit}


\usepackage[margin=2cm]{geometry}
\usepackage{eso-pic}         % Pour ajouter des éléments en arrière-plan

% Commande pour ajouter du texte en arrière-plan
\AddToShipoutPicture{
    \AtTextCenter{%
        \makebox[0pt]{\rotatebox{45}{\textcolor[gray]{0.9}{\fontsize{5cm}{5cm}\selectfont Pathé BA}}}
    }
}

\begin{document}

\begin{minipage}{0.8\textwidth}
	Talla                        
\end{minipage}
\begin{minipage}{0.8\textwidth}
	Diallo 
\end{minipage}

\begin{center}
\textbf{{\underline{\textcolor{green}{Correction}}}}
\end{center}
\section*{\textcolor{green}{\underline{Exercice 2}:}}

Le tableau suivant indique, pour l'année \(X_i\), la quantité \(Y_i\) (exprimée en millions de mètres cubes) d'essence consommée dans un pays donné.

\begin{tabular}{|c|c|c|c|c|c|c|c|c|c|c|}
  \hline
  $X_{i}$ & 2005 & 2006 & 2007 & 2008 & 2009 & 2010 & 2011 & 2012 & 2013 & 2014 \\
  \hline
  $Y_{i}$ & 6 & 6,7 & 7,2 & 8,1 & 8,5 & 9 & 9,7 & 10,3 & 10,9 & 11,6 \\
  \hline
\end{tabular}
On pose : X=Années(1 à 10) et Y=Quantité d'essence consommée

On suppose que le nuage de points associé aux variable \(Y\) et \(X\) présente une forme linéaire et on se propose d'étudier la régression et la corrélation linéaire \(Y\) et \(X\).
\begin{enumerate}
  \item Représenter le nuage de points
  \item Déterminer la droite de régression linéaire \(Y\) en \(X\) et représenter cette droite sur le nuage de points.
  \item Calculer le coefficient de corrélation linéaire \(R\) entre \(X\) et \(Y\) et interpréter la qualité de corrélation linéaire.
  \item Quelle consommation pourrait-on prévoir en 2016 ?
  \item En quelle année pourrait-on prévoir une consommation de 15,1 millions de mètres cubes d'essence ?
\end{enumerate}
\section*{Solutions}

\subsection*{1. Nuage de Points}
Le nuage de points peut être représenté en utilisant un graphique, comme indiqué ci-dessous :
\newpage
\begin{figure}[h]
    \centering
   \includegraphics[width=0.7\textwidth]{nuagePoints.png} % Remplacez par le nom de votre image
    \caption{Nuage de points représentant la consommation d'essence}
    \label{fig:nuage_points}
\end{figure}

\section*{Détermination de la droite de régression linéaire \(Y\) en \(X\)}

Nous allons déterminer la droite de régression linéaire \(Y\) en fonction de \(X\) à partir des données suivantes :

\begin{table}[h]
    \centering
    \begin{tabular}{|c|c|c|c|c|c|c|c|c|c|c|}
        \hline
        \(X_{i}\) & 2005 & 2006 & 2007 & 2008 & 2009 & 2010 & 2011 & 2012 & 2013 & 2014 \\
        \hline
        \(Y_{i}\) & 6 & 6.7 & 7.2 & 8.1 & 8.5 & 9 & 9.7 & 10.3 & 10.9 & 11.6 \\
        \hline
    \end{tabular}
\end{table}

La formule de la régression linéaire s'exprime sous la forme :

\[
Y = aX + b
\]

où \(a\) est la pente de la droite et \(b\) est l'ordonnée à l'origine.

\subsection*{1. Calcul des moyennes \(\bar{X}\) et \(\bar{Y}\)}

\[
\bar{X} = \frac{1}{n} \sum_{i=1}^{n} X_i = \frac{2005 + 2006 + 2007 + 2008 + 2009 + 2010 + 2011 + 2012 + 2013 + 2014}{10} = 2009.5
\]

\[
\bar{Y} = \frac{1}{n} \sum_{i=1}^{n} Y_i = \frac{6 + 6.7 + 7.2 + 8.1 + 8.5 + 9 + 9.7 + 10.3 + 10.9 + 11.6}{10} = 9.16
\]

\subsection*{2. Calcul de \(a\)}

Calculons les termes nécessaires pour \(a\):

\[
\sum (X_i - \bar{X})(Y_i - \bar{Y}), \quad \sum (X_i - \bar{X})^2
\]

\begin{table}[h]
    \centering
    \begin{tabular}{|c|c|c|c|c|c|}
        \hline
        \(X_i\) & \(Y_i\) & \(X_i - \bar{X}\) & \(Y_i - \bar{Y}\) & \((X_i - \bar{X})(Y_i - \bar{Y})\) & \((X_i - \bar{X})^2\) \\
        \hline
        2005 & 6 & -4.5 & -3.16 & 14.22 & 20.25 \\
        2006 & 6.7 & -3.5 & -2.46 & 8.61 & 12.25 \\
        2007 & 7.2 & -2.5 & -1.96 & 4.90 & 6.25 \\
        2008 & 8.1 & -1.5 & -1.06 & 1.59 & 2.25 \\
        2009 & 8.5 & -0.5 & -0.66 & 0.33 & 0.25 \\
        2010 & 9 & 0.5 & -0.16 & -0.08 & 0.25 \\
        2011 & 9.7 & 1.5 & 0.54 & 0.81 & 2.25 \\
        2012 & 10.3 & 2.5 & 1.14 & 2.85 & 6.25 \\
        2013 & 10.9 & 3.5 & 1.74 & 6.09 & 12.25 \\
        2014 & 11.6 & 4.5 & 2.44 & 10.98 & 20.25 \\
        \hline
        \textbf{Total} & & & & \textbf{49.76} & \textbf{92.50} \\
        \hline
    \end{tabular}
\end{table}

Calculons \(a\):

\[
a = \frac{49.76}{92.50} \approx 0.538
\]

\subsection*{3. Calcul de \(b\)}

\[
b = \bar{Y} - a\bar{X} = 9.16 - 0.538 \times 2009.5 \approx -1011.1
\]

\subsection*{Résultat final}

La droite de régression linéaire est :

\[
Y = 0.538X - 1011.1
\]

Cette équation peut être utilisée pour prédire les valeurs de \(Y\) en fonction de \(X\).

\section*{Calcul du coefficient de corrélation linéaire \(R\)}

Pour calculer le coefficient de corrélation linéaire \(R\) entre \(X\) et \(Y\), nous utilisons la formule :

\[
R = \frac{\sum_{i=1}^{n} (X_i - \bar{X})(Y_i - \bar{Y})}{\sqrt{\sum_{i=1}^{n} (X_i - \bar{X})^2} \sqrt{\sum_{i=1}^{n} (Y_i - \bar{Y})^2}}
\]

Nous avons déjà calculé :

\[
\bar{X} = 2009.5, \quad \bar{Y} = 9.16
\]

Nous savons que :

\[
\sum (X_i - \bar{X})(Y_i - \bar{Y}) = 49.76
\]

Calculons \(\sum (Y_i - \bar{Y})^2\) :

\begin{table}[h]
    \centering
    \begin{tabular}{|c|c|c|}
        \hline
        \(Y_i\) & \(Y_i - \bar{Y}\) & \((Y_i - \bar{Y})^2\) \\
        \hline
        6 & -3.16 & 10.0256 \\
        6.7 & -2.46 & 6.0516 \\
        7.2 & -1.96 & 3.8416 \\
        8.1 & -1.06 & 1.1236 \\
        8.5 & -0.66 & 0.4356 \\
        9 & -0.16 & 0.0256 \\
        9.7 & 0.54 & 0.2916 \\
        10.3 & 1.14 & 1.2996 \\
        10.9 & 1.74 & 3.0276 \\
        11.6 & 2.44 & 5.9536 \\
        \hline
        \textbf{Total} & & \textbf{41.0752} \\
        \hline
    \end{tabular}
\end{table}

Ainsi,

\[
\sum (Y_i - \bar{Y})^2 = 41.0752
\]

Calculons maintenant \(R\) :

\[
R = \frac{49.76}{\sqrt{92.50} \cdot \sqrt{41.0752}} = \frac{49.76}{9.624 \cdot 6.403} \approx \frac{49.76}{61.66} \approx 0.807
\]

\subsection*{Interprétation de \(R\)}

La valeur de \(R\) est de **0.807**, ce qui indique une **corrélation positive forte** entre \(X\) et \(Y\). Cela signifie que lorsque \(X\) augmente, \(Y\) a tendance à augmenter également, indiquant une tendance claire dans les données.

\section*{Prévision de la consommation en 2016}

Pour prévoir la consommation en 2016, nous utilisons l'équation de la droite de régression :

\[
Y = 0.538X - 1011.1
\]

En substituant \(X = 2016\) dans l'équation :

\[
Y = 0.538 \times 2016 - 1011.1
\]

Calculons :

\[
Y = 0.538 \times 2016 \approx 1084.368
\]
\[
Y \approx 1084.368 - 1011.1 \approx 73.268
\]

Ainsi, la consommation prévue pour l'année 2016 est d'environ **73.27** (selon les unités de \(Y\) utilisées).
\section*{Prévision de l'année pour une consommation de 15,1 millions de mètres cubes}

Pour déterminer l'année à laquelle la consommation de 15,1 millions de mètres cubes d'essence pourrait être atteinte, nous utilisons l'équation de la droite de régression :

\[
Y = 0.538X - 1011.1
\]

En substituant \(Y = 15.1\) dans l'équation :

\[
15.1 = 0.538X - 1011.1
\]

En ajoutant \(1011.1\) des deux côtés :

\[
15.1 + 1011.1 = 0.538X
\]

\[
1026.2 = 0.538X
\]

En divisant par \(0.538\) :

\[
X = \frac{1026.2}{0.538} \approx 1908.15
\]

Ainsi, l'année prévue pour atteindre une consommation de **15,1 millions de mètres cubes** d'essence est **1908**.

\end{document}