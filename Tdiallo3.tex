\documentclass{article}
\usepackage{amsmath}
\begin{document}

\section*{Exercice 3}

On jette une pièce de monnaie 3 fois de suite.

\begin{enumerate}
    \item Donner la liste de tous les résultats possibles en notant P pour Pile et F pour Face (exemple : PPF).
    \item Donner la probabilité des événements suivants :
    \begin{enumerate}
        \item A : \text{le tirage ne comporte que des Piles}.
        \item B : \text{le tirage comporte au moins une fois Face}.
    \end{enumerate}
\end{enumerate}

\section*{Correction}

On jette une pièce de monnaie 3 fois de suite.

\begin{enumerate}
    \item Donner la liste de tous les résultats possibles en notant P pour Pile et F pour Face (exemple : PPF).
    
    \textbf{Résultats possibles :} \\
    PPP, PPF, PFP, PFF, FPP, FPF, FFPP, FFF

    \item Donner la probabilité des événements suivants :
    \begin{enumerate}
        \item A : \text{le tirage ne comporte que des Piles}.
        
        \textbf{Événement A :} \\
        Le seul résultat favorable est PPP. \\
        \text{La probabilité de l'événement A est } $P(A) = \frac{1}{8}$ \text{ (il y a 8 résultats possibles)}.

        \item B : \text{le tirage comporte au moins une fois Face}.
        
        \textbf{Événement B :} \\
        Les résultats qui ne comportent pas de Face sont uniquement PPP. \\
        Par conséquent, les résultats comportant au moins une Face sont : PPF, PFP, PFF, FPP, FPF, FFP, FFF. \\
        Le nombre de résultats favorables est 7. \\
        \text{La probabilité de l'événement B est } $P(B) = \frac{7}{8}$.
    \end{enumerate}
\end{enumerate}
\end{document}