\documentclass[12pt,a4paper]{article}
\usepackage{amsmath,amssymb,mathrsfs,tikz,times,pifont}
\usepackage{enumitem}
\newcommand\circitem[1]{%
\tikz[baseline=(char.base)]{
\node[circle,draw=gray, fill=red!55,
minimum size=1.2em,inner sep=0] (char) {#1};}}
\newcommand\boxitem[1]{%
\tikz[baseline=(char.base)]{
\node[fill=cyan,
minimum size=1.2em,inner sep=0] (char) {#1};}}
\setlist[enumerate,1]{label=\protect\circitem{\arabic*}}
\setlist[enumerate,2]{label=\protect\boxitem{\alph*}}
%%%::::::by chnini ameur :::::::%%%
\everymath{\displaystyle}
\usepackage[left=1cm,right=1cm,top=1cm,bottom=1.7cm]{geometry}
\usepackage{array,multirow}
\usepackage[most]{tcolorbox}
\usepackage{varwidth}
\tcbuselibrary{skins,hooks}
\usetikzlibrary{patterns}
%%%::::::by chnini ameur :::::::%%%
\newtcolorbox{exa}[2][]{enhanced,breakable,before skip=2mm,after skip=5mm,
colback=yellow!20!white,colframe=black!20!blue,boxrule=0.5mm,
attach boxed title to top left ={xshift=0.6cm,yshift*=1mm-\tcboxedtitleheight},
fonttitle=\bfseries,
title={#2},#1,
% varwidth boxed title*=-3cm,
boxed title style={frame code={
\path[fill=tcbcolback!30!black]
([yshift=-1mm,xshift=-1mm]frame.north west)
arc[start angle=0,end angle=180,radius=1mm]
([yshift=-1mm,xshift=1mm]frame.north east)
arc[start angle=180,end angle=0,radius=1mm];
\path[left color=tcbcolback!60!black,right color = tcbcolback!60!black,
middle color = tcbcolback!80!black]
([xshift=-2mm]frame.north west) -- ([xshift=2mm]frame.north east)
[rounded corners=1mm]-- ([xshift=1mm,yshift=-1mm]frame.north east)
-- (frame.south east) -- (frame.south west)
-- ([xshift=-1mm,yshift=-1mm]frame.north west)
[sharp corners]-- cycle;
},interior engine=empty,
},interior style={top color=yellow!5}}
%%%%%%%%%%%%%%%%%%%%%%%

\usepackage{fancyhdr}
\usepackage{eso-pic}         % Pour ajouter des éléments en arrière-plan
% Commande pour ajouter du texte en arrière-plan
\AddToShipoutPicture{
    \AtTextCenter{%
        \makebox[0pt]{\rotatebox{80}{\textcolor[gray]{0.7}{\fontsize{5cm}{5cm}\selectfont PGB}}}
    }
}
\usepackage{lastpage}
\fancyhf{}
\pagestyle{fancy}
\renewcommand{\footrulewidth}{1pt}
\renewcommand{\headrulewidth}{0pt}
\renewcommand{\footruleskip}{10pt}
\fancyfoot[R]{
\color{blue}\ding{45}\ \textbf{2024}
}
\fancyfoot[L]{
\color{blue}\ding{45}\ \textbf{Prof:M. Diallo}
}
\cfoot{\bf
\thepage /
\pageref{LastPage}}
\begin{document}
\renewcommand{\arraystretch}{1.5}
\renewcommand{\arrayrulewidth}{1.2pt}
\begin{tikzpicture}[overlay,remember picture]
\node[draw=blue,line width=1.2pt,fill=purple,text=blue,inner sep=3mm,rounded corners,pattern=dots]at ([yshift=-2.5cm]current page.north) {\begingroup\setlength{\fboxsep}{0pt}\colorbox{white}{\begin{tabular}{|*1{>{\centering \arraybackslash}p{0.28\textwidth}} |*2{>{\centering \arraybackslash}p{0.2\textwidth}|} *1{>{\centering \arraybackslash}p{0.19\textwidth}|} }
\hline
\multicolumn{3}{|c|}{$\diamond$$\diamond$$\diamond$\ \textbf{Lycée de Dindéfélo}\ $\diamond$$\diamond$$\diamond$ }& \textbf{A.S. : 2024/2025} \\ \hline
\textbf{Matière: Mathématiques}& \textbf{Niveau : $1^{er}$}\textbf{L} &\textbf{Date: 09/12/2024} & \textbf{Durée : 2 heures} \\ \hline
\multicolumn{4}{|c|}{\parbox[c]{10cm}{\begin{center}
\textbf{{\Large\sffamily Devoir n$ ^{\circ} $ 1 Du 1$ ^\text{\bf er} $ Semestre}}
\end{center}}} \\ \hline
\end{tabular}}\endgroup};
\end{tikzpicture}
\vspace{3cm}

\section*{Questions de Cours : 4 pts}
Complète les pointillés :
\begin{enumerate}
    \item On appelle racine réelle d’un polynôme \( P \) tout nombre réel \( \alpha \) tel que \ldots \hfill (1pt)
    \item Si un polynôme \( P \) a une racine réelle \( \alpha \), alors il existe un polynôme \( Q \) tel que \( P(x) = \ldots \) \hfill (1pt)
    \item Soit \( P(x) = ax^2 + bx + c \) un trinôme du second degré.
        \begin{enumerate}
            \item Si \( P(x) \) admet deux racines \( x_1 \) et \( x_2 \) distinctes, alors sa forme factorisée est \( P(x) = \ldots \) \hfill (1pt)
            \item Si \( P(x) \) admet une racine double \( x_0 \), alors sa forme factorisée est \( P(x) = \ldots \) \hfill (1pt)
        \end{enumerate}
\end{enumerate}
\section*{\underline{Exercice 1 :} $5$ points}

Résoudre dans \( \mathbb{R}^3 \) les systèmes d’équations suivants :
\begin{enumerate}
    \item \(
    \begin{aligned}
        &\begin{cases}
            x + y + z = 8 \\
            \quad\quad y - z = 1 \\
            \quad\quad\quad 4z = 8
        \end{cases} \quad &\textbf{(1,5pt)}
    \end{aligned}
    \)
    \item \(
    \begin{aligned}
        &\begin{cases}
            \quad\quad\quad 3z = 6 \\
            \quad\quad y + z = 4 \\
            2x + y - z = 8
        \end{cases} \quad &\textbf{(1,5pt)}
    \end{aligned}
    \)
    \item \(
    \begin{aligned}
        &\begin{cases}
            2x - y + 2z = -4 \\
            x + y + 2z = -1 \\
            4x + y + 4z = -2
        \end{cases} \quad &\textbf{(2pt)}
    \end{aligned}
    \)
\end{enumerate}


\section*{\underline{Exercice 1 :} $10$ points}
Soit le polynôme \( f(x) = x^3 + ax^2 + bx + 6 \) où \( a \) et \( b \) sont deux nombres réels.

\begin{enumerate}
    \item Déterminer \( a \) et \( b \) sachant que \( f(-2) = 0 \) et \( f(-1) = 8 \). \hfill (2pts)
    \item On pose \( f(x) = x^3 - 2x^2 - 5x + 6 \).
        \begin{enumerate}
            \item Calculer \( f(1) \). Que peut-on en déduire ? \hfill (0.5 + 0.5)pt
            \item Déterminer un polynôme \( Q(x) \) tel que \( f(x) = (x - 1)Q(x) \). \hfill (2pts)
            \item Factoriser \( Q(x) \) et en déduire une factorisation complète de \( f(x) \). \hfill (1 + 0.5)pt
            \item Résoudre dans \( \mathbb{R} \) :
            \begin{itemize}
                \item \( f(x) = 0 \) \hfill (1.5pt)
                \item \( f(x) \geq 0 \) \hfill (2pts)
            \end{itemize}
        \end{enumerate}
\end{enumerate}
\end{document}