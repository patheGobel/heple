\documentclass[12pt,a4paper]{article}
\usepackage{array,amsmath,amssymb,mathrsfs, makecell,tikz,times,pifont}
\usepackage{enumitem}

\newcommand\circitem[1]{%
\tikz[baseline=(char.base)]{
\node[circle,draw=gray, fill=red!55,
minimum size=1.2em,inner sep=0] (char) {#1};}}
\newcommand\boxitem[1]{%
\tikz[baseline=(char.base)]{
\node[fill=cyan,
minimum size=1.2em,inner sep=0] (char) {#1};}}
\setlist[enumerate,1]{label=\protect\circitem{\arabic*}}
\setlist[enumerate,2]{label=\protect\boxitem{\alph*}}
%%%::::::by chnini ameur :::::::%%%
\everymath{\displaystyle}
\usepackage[left=1cm,right=1cm,top=1cm,bottom=1.7cm]{geometry}
\usepackage[colorlinks=true, linkcolor=blue, urlcolor=blue, citecolor=blue]{hyperref}
\usepackage{array,multirow}
\usepackage[most]{tcolorbox}
\usepackage{varwidth}
\usepackage{float} %pour utiliser l'option [H] qui force l'image à apparaître exactement à l'endroit où elle est placée dans le code.
\tcbuselibrary{skins,hooks}
\usetikzlibrary{patterns}
%%%::::::by chnini ameur :::::::%%%
\newtcolorbox{exa}[2][]{enhanced,breakable,before skip=2mm,after skip=5mm,
colback=yellow!20!white,colframe=black!20!blue,boxrule=0.5mm,
attach boxed title to top left ={xshift=0.6cm,yshift*=1mm-\tcboxedtitleheight},
fonttitle=\bfseries,
title={#2},#1,
% varwidth boxed title*=-3cm,
boxed title style={frame code={
\path[fill=tcbcolback!30!black]
([yshift=-1mm,xshift=-1mm]frame.north west)
arc[start angle=0,end angle=180,radius=1mm]
([yshift=-1mm,xshift=1mm]frame.north east)
arc[start angle=180,end angle=0,radius=1mm];
\path[left color=tcbcolback!60!black,right color = tcbcolback!60!black,
middle color = tcbcolback!80!black]
([xshift=-2mm]frame.north west) -- ([xshift=2mm]frame.north east)
[rounded corners=1mm]-- ([xshift=1mm,yshift=-1mm]frame.north east)
-- (frame.south east) -- (frame.south west)
-- ([xshift=-1mm,yshift=-1mm]frame.north west)
[sharp corners]-- cycle;
},interior engine=empty,
},interior style={top color=yellow!5}}
%%%%%%%%%%%%%%%%%%%%%%%

\usepackage{fancyhdr}
%\usepackage{eso-pic}         % Pour ajouter des éléments en arrière-plan
% Commande pour ajouter du texte en arrière-plan
\usepackage{tkz-tab}
%\AddToShipoutPicture{
   % \AtTextCenter{%
      %  \makebox[0pt]{\rotatebox{35}{\textcolor[gray]{0.5}{\fontsize{1.5cm}{1.5cm}\selectfont DIONE@MATHS TL}}}
%    }
%}
\usepackage{lastpage}
\fancyhf{}
\pagestyle{fancy}
\renewcommand{\footrulewidth}{1pt}
\renewcommand{\headrulewidth}{0pt}
\renewcommand{\footruleskip}{10pt}
\fancyfoot[R]{
\color{red}\ding{45}\ \textbf{TL, 2025}
}
\fancyfoot[L]{
\color{black}\ding{45}\ \textbf{Prof : M. DIONE}
}
\cfoot{\bf
\thepage /
\pageref{LastPage}}
\begin{document}
\renewcommand{\arraystretch}{1.5}
\renewcommand{\arrayrulewidth}{1.2pt}
\begin{tikzpicture}[overlay,remember picture]
\node[draw=black,line width=1.2pt,fill=purple,text=black,inner sep=3mm,rounded corners,pattern=dots]at ([yshift=-2.5cm]current page.north) {\begingroup\setlength{\fboxsep}{0pt}\colorbox{white}{\begin{tabular}{|*1{>{\centering \arraybackslash}p{0.28\textwidth}} |*2{>{\centering \arraybackslash}p{0.2\textwidth}|} *1{>{\centering \arraybackslash}p{0.19\textwidth}|} }
\hline
\multicolumn{3}{|c|}{$\star$$\star$$\star$\ \textbf{Lycée ex CEM de Koumbal}\ $\star$$\star$$\star$ }& \textbf{A.S. : 2024/2025} \\ \hline

\multicolumn{4}{|c|}{\parbox[c]{10cm}{\begin{center}
\textbf{{\Large\sffamily DEVOIR N°1 DU SECOND SEMESTRE}}
\end{center}}} \\ \hline
\end{tabular}}\endgroup};
\end{tikzpicture}
\vspace{3cm}

\section*{\underline{Exercice 1}: (10 pts)}
Dans les affirmations suivantes, l'élève choisira une seule parmi les réponses proposées (\textit{chaque bonne reponse t'apportera 1 pt}).\\

\renewcommand{\arraystretch}{2} % Pour aérer un peu les lignes du tableau

\raggedright
\begin{tabular}{|c|p{6cm}|p{3.2cm}|p{3.2cm}|p{4.2cm}|}
\hline
N° & Affirmations & Réponse A & Réponse B & Réponse C \\
\hline
1 & $\lim\limits_{x \to 0} x^4 \ln(x) =$ & $+\infty$ & $0$ & $-\infty$ \\
\hline
2 & $\lim\limits_{x \to +\infty} e^x = $ & $0$ & $ e$ & $+\infty$  \\
\hline
3 & \makecell[l]{L'équation $\ln(3x-4) = \ln(2x - 1)$\\ a pour solution :} & S = \{1\} & S = \{-3\} & S = \{3\}  \\
\hline
4 & \makecell[l]{L'inéquation\\ $\ln(x^2 +1)\leq \ln(2x^2 + x+ 2) $\\ 
  a pour solution} & S = $\emptyset$ & S = $]-\infty ; 0]$ & S = [0 ; 1]  \\
\hline 
5 & $\lim\limits_{x \to +\infty} \frac{\ln(x)}{x} = $ & $0$ & $+\infty$ & $-\infty $  \\
\hline
6 & \makecell[l]{$f(x) = \ln\left( \frac{x-1}{x}\right) $ a pour\\ domaine de définition} & $D_f = \mathbb{R}$  & $D_f = \mathbb{R^*}$ & $D_f  = ]-\infty ;0[ \cup ]1 ; +\infty[$  \\
\hline
7 &  $\lim\limits_{x \to 0} \frac{e^x -1}{x}=$ & 1 & $\frac{1}{2}$ & $0$  \\
\hline
8 & \makecell[l]{Le domaine de la fonction\\ $f(x)= x+3e^x $ est :} & $D_f = \mathbb{R}$ & $D_f = \mathbb{R^*} $ & $D_f = ]0 ; +\infty[ $  \\
\hline 
9 & \makecell[l]{La dérivée de la fonction\\ $f(x) = \ln(x-e)$ est :}  & $f'(x) = \frac{e}{x-1}$ &  $f'(x) = \frac{-1}{x-e}$ &  $f'(x) = \frac{1}{x-e}$ \\
\hline
10 & La dérivée de la fonction $f(x) = \frac{e^x}{x}$ est : & $f'(x) = \frac{e^x(x-1)}{x^2}$  & $f'(x)= \frac{e^x(x+1)}{x^2}$  & $f'(x)= \frac{e^x(x-1)}{x}$\\
\hline
\end{tabular}

\section*{\underline{Exercice 2}: (05 pts)}
Soit $f(x) = x^{3} -2x^{2}-x+2 $
\begin{enumerate}

\item Calculer $f(1)$ puis factorise complètement $f(x)$  (\textbf{1 pt})
\item Résoudre $(ln(x))^{3}-2(ln(x))^{2} -ln(x)+2 =0$   (\textbf{0.5 pt})
\item {Résoudre $ f(ln(x+2))= 0 $ }   (\textbf{0.5 pt})
\item Résoudre dans $\mathbb{R}^{2}$:
\[
\begin{cases}
x+y=3\\
ln(x)+ln(y)=ln(2)
\end{cases}  (\textbf{1 pt})
\] 

\item Développer $x(x-1)(x-2)$   (\textbf{0.5 pt})
\item Résoudre $(2ln(x))^{3}-6(ln(x))^{2} +4ln(x)=0$   (\textbf{0.5 pt})
\item $2e^{2x}+4e^{y}+2=0$    (\textbf{1 pt})
\item Résoudre dans $\mathbb{R}^{2}$ : 
\[
\begin{cases}
e^{x}-2e^{y}=-5\\
3e^{x}+e^{y}=13 
\end{cases}  (\textbf{0.5 pt})
\]

\end{enumerate}


\section*{\underline{Problème}: (5 pts)}
Soit $f$ la foncion définie par ; $f(x)= \ln\left(\frac{2x+2}{x-1}\right)$
\begin{enumerate}
\item Déterminer $Df$, les limites aux bornes de $Df$ et les éventuelles asymptotes (\textbf{1 pt})
\item Drésser le tableau de variation de $f$   (\textbf{1 pt})
\item Dterminer une équation de la tangente en $A$ intersection de $(Cf)$ et l'axe $(xx')$  (\textbf{0.5 pt})
\item $(Cf)$ rencontre -t-elle $(yy')$ ? justifie ta réponse.     (\textbf{0.5 pt})
\item Montrer  $\Omega(0;\ln(2))$ est un centre de symétrie de $(Cf)$.   (\textbf{1 pt})
\item Tracer $(Cf)$ dans un repère orthonormé directe.    (\textbf{1 pt})
\end{enumerate}

\end{document}