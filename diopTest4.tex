\documentclass[a4paper,12pt]{article}
\usepackage{graphicx}
\usepackage[a4paper, top=0cm, bottom=2cm, left=2cm, right=2cm]{geometry} % Ajuste les marges
\usepackage{xcolor} % Pour ajouter des couleurs
\usepackage{hyperref} % Pour avoir des références colorées si nécessaire
\usepackage{eso-pic}         % Pour ajouter des éléments en arrière-plan

\usepackage[french]{babel}
\usepackage[T1]{fontenc}
\usepackage{mathrsfs}
\usepackage{amsmath}
\usepackage{amsfonts}
\usepackage{amssymb}
\usepackage{tkz-tab}

\usepackage{tikz}
\usetikzlibrary{arrows, shapes.geometric, fit}
\newcounter{correction} % Compteur pour les questions

% Définir la commande pour afficher une question numérotée
\newcommand{\question}{%
  \refstepcounter{correction}%
  \textbf{\textcolor{black}{Question \thecorrection (1 point) :}} \ignorespaces
}

\begin{document}
\hrule % Barre horizontale
% En-tête
\begin{center}
    \begin{tabular}{@{} p{5cm} p{5cm} p{5cm} @{}} % 3 colonnes avec largeurs fixées
        Lycée Maciré BA & Test 4 & 04 Octobre 2024 \\
    \end{tabular}
    \\[-0.01cm] % Ajuster l'espace vertical entre le tableau et la barre
    \hrule % Barre horizontale
\end{center}
\begin{center}
    \textbf{\Large Limites et Continuité} \\[0.2cm]
    \textbf{\large Professeur : M. DIOP} \\[0.2cm]
    \textbf{Classe : Terminale S2} \\[0.2cm]
    \textbf{\small Durée : 10 minutes} \\[0.2cm]
    \textbf{\small Note :\quad\quad /5}
\end{center}

% Nom de l'élève
\textbf{\small Nom de l'élève :} \underline{\hspace{8cm}} \\[0.5cm]

% Introduction aux questions
Complétez les exercices suivants en utilisant le cours et vos connaissances sur la continuité des fonctions. \\[0.3cm]

\question Enoncer le théorème des valeurs intermediaries.\\
\underline{\hspace{20cm}}\\
\underline{\hspace{20cm}}\\
\underline{\hspace{20cm}}\\
\question Enoncer le théorème d’existence et d’unicité d’une solution\\
\underline{\hspace{20cm}}\\
\underline{\hspace{20cm}}\\
\underline{\hspace{20cm}}\\
\question\
Complétez la phrase suivante : Une fonction est dite continue sur un intervalle \( I \) si elle est continue \underline{\hspace{15cm}}. \\[0.3cm]
\underline{\hspace{20cm}}

\question\\
\[\text{Si}\lim_{x \to -\infty} h(x) =-\infty \text{ et }\lim_{x \to -\infty}\frac{h(x)}{x}= 0\text{ alors } (C_{h}) \underline{\hspace{10cm}}\]\\
\underline{\hspace{20cm}}\\[0.3cm]
\[\text{Si}\lim_{x \to +\infty} h(x) =+\infty \text{ et }\lim_{x \to +\infty}\frac{h(x)}{x}=\gamma \in\mathbb{R}^{*}\text{ et }\lim_{x \to +\infty}[h(x)-\gamma x]=+\infty\text{ alors } \underline{\hspace{10cm}}\]\\
\underline{\hspace{20cm}}\\[0.3cm]
\question\\
Soit \( f(x) =\frac{2x+1}{x-1} \) une fonction continue décroissante sur \( \mathbb{R}\setminus\{1\} \) déterminer:\\
\( f([-2; 0])=\underline{\hspace{15cm}} \)\\
\( f(]1; +\infty[)=\underline{\hspace{15cm}} \)

\end{document}