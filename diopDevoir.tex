\documentclass[12pt,a4paper]{article}
\usepackage{amsmath,amssymb,mathrsfs,tikz,times,pifont}
\usepackage{enumitem}
\newcommand\circitem[1]{%
\tikz[baseline=(char.base)]{
\node[circle,draw=gray, fill=red!55,
minimum size=1.2em,inner sep=0] (char) {#1};}}
\newcommand\boxitem[1]{%
\tikz[baseline=(char.base)]{
\node[fill=cyan,
minimum size=1.2em,inner sep=0] (char) {#1};}}
\setlist[enumerate,1]{label=\protect\circitem{\arabic*}}
\setlist[enumerate,2]{label=\protect\boxitem{\alph*}}
%%%::::::by chnini ameur :::::::%%%
\everymath{\displaystyle}
\usepackage[left=1cm,right=1cm,top=1cm,bottom=1.7cm]{geometry}
\usepackage{array,multirow}
\usepackage[most]{tcolorbox}
\usepackage{varwidth}
\tcbuselibrary{skins,hooks}
\usetikzlibrary{patterns}
%%%::::::by chnini ameur :::::::%%%
\newtcolorbox{exa}[2][]{enhanced,breakable,before skip=2mm,after skip=5mm,
colback=yellow!20!white,colframe=black!20!blue,boxrule=0.5mm,
attach boxed title to top left ={xshift=0.6cm,yshift*=1mm-\tcboxedtitleheight},
fonttitle=\bfseries,
title={#2},#1,
% varwidth boxed title*=-3cm,
boxed title style={frame code={
\path[fill=tcbcolback!30!black]
([yshift=-1mm,xshift=-1mm]frame.north west)
arc[start angle=0,end angle=180,radius=1mm]
([yshift=-1mm,xshift=1mm]frame.north east)
arc[start angle=180,end angle=0,radius=1mm];
\path[left color=tcbcolback!60!black,right color = tcbcolback!60!black,
middle color = tcbcolback!80!black]
([xshift=-2mm]frame.north west) -- ([xshift=2mm]frame.north east)
[rounded corners=1mm]-- ([xshift=1mm,yshift=-1mm]frame.north east)
-- (frame.south east) -- (frame.south west)
-- ([xshift=-1mm,yshift=-1mm]frame.north west)
[sharp corners]-- cycle;
},interior engine=empty,
},interior style={top color=yellow!5}}
%%%%%%%%%%%%%%%%%%%%%%%
\usepackage{fancyhdr}
\usepackage{lastpage}
\fancyhf{}
\pagestyle{fancy}
\renewcommand{\footrulewidth}{1pt}
\renewcommand{\headrulewidth}{0pt}
\renewcommand{\footruleskip}{10pt}
\fancyfoot[R]{
\color{blue}\ding{45}\ \textbf{2024}
}
\fancyfoot[L]{
\color{blue}\ding{45}\ \textbf{Prof:M. Diop}
}
\cfoot{\bf
\thepage /
\pageref{LastPage}}
\begin{document}
\renewcommand{\arraystretch}{1.5}
\renewcommand{\arrayrulewidth}{1.2pt}
\begin{tikzpicture}[overlay,remember picture]
\node[draw=blue,line width=1.2pt,fill=purple,text=blue,inner sep=3mm,rounded corners,pattern=dots]at ([yshift=-2.5cm]current page.north) {\begingroup\setlength{\fboxsep}{0pt}\colorbox{white}{\begin{tabular}{|*1{>{\centering \arraybackslash}p{0.28\textwidth}} |*2{>{\centering \arraybackslash}p{0.2\textwidth}|} *1{>{\centering \arraybackslash}p{0.19\textwidth}|} }
\hline
\multicolumn{3}{|c|}{$\diamond$$\diamond$$\diamond$\ \textbf{Lycée Maciré BA}\ $\diamond$$\diamond$$\diamond$ }& \textbf{A.S. : 2024/2025} \\ \hline
\textbf{Matière: Mathématiques}& \textbf{Niveau : 1}$ ^\text{\bf er} $\textbf{S2} &\textbf{Date: 03/12/2024} & \textbf{Durée : 4 heures} \\ \hline
\multicolumn{4}{|c|}{\parbox[c]{10cm}{\begin{center}
\textbf{{\Large\sffamily Devoir n$ ^{\circ} $ 1 Du 1$ ^\text{\bf er} $ Semestre}}
\end{center}}} \\ \hline
\end{tabular}}\endgroup};
\end{tikzpicture}
\vspace{3cm}
\section*{\underline{Exercice n°1 : (04,5pts)}}
Résoudre les équations et inéquations suivantes :
\begin{enumerate}
    \item $-x^4 + 5x^2 - 6 \leq 0$
    \item $\sqrt{2x^2 - 7x + 4} = \sqrt{-x^2 - 3x + 4}$
    \item $x^2 - 3x + \sqrt{x^2 - 3x + 11} = 1$
    \item $\sqrt{11 - x} = \sqrt{5x + 15} - \sqrt{3x - 2}$
    \item $2\sqrt{x - x^2} \leq 2x - 1$
    \item $\sqrt{x^2 + 3x - 4} > -x + 3$
\end{enumerate}
\section*{\underline{Exercice n°2 : (04,5pts)}}

Soit $P(x) = x^4 - 12x^3 + 37x^2 - 12x + 1$
\begin{enumerate}
    \item Montrer que $0$ n'est pas racine.
    \item Montrer que si $\alpha$ est racine de $P(x)$, alors $\frac{1}{\alpha}$ l'est aussi.
    \item Montrer que :
    \[
    P(x) = x^2 \left[ \left( x + \frac{1}{x} \right)^2 - 12 \left( x + \frac{1}{x} \right) + 35 \right]
    \]
    \item Résoudre dans $\mathbb{R}$ $P(x) = 0$ puis $P(x) \geq 0$.
\end{enumerate}
\section*{\underline{Exercice n°3 : (03pts)}}

\begin{enumerate}

\item Résoudre dans \( \mathbb{R}^3 \) le système :
\[
\begin{cases}
-x + 2y + z = 1, \\
-2x + 3y + z = 1, \\
4x + y + 2z = 8.
\end{cases}
\]

\item Résoudre dans \( \mathbb{R}^2 \) les systèmes :

\[
\text{a) }
\begin{cases}
x^2 + y^2 = 65 \\
x + y = -3
\end{cases}
\quad
\quad
\text{b) }
\begin{cases}
x + y = 1 \\
x^3 + y^3 = 7
\end{cases}
\]
\end{enumerate}
\section*{\underline{Exercice n°4 : (03pts)}}
Soit l'équation : $(E_m) : (m-2)x^2 - 2(m-1)x + m - 3 = 0.$
\begin{enumerate}
    \item Étudier l'existence et le signe des racines.
    \item Déterminer une relation indépendante de $m$ liant les racines $x_1$ et $x_2$ au cas où elles existent.
\end{enumerate}

\section*{\underline{Exercice n°5 : (03pts)}}

\bigskip

\textbf{I.} Soit \( P(x) = x^3 - bx^2 + a(2b - a)x - a^2b \) avec \( a \) et \( b \) deux réels.

\begin{enumerate}
    \item Montrer que \( P(x) \) est divisible par \( x - a \).
    \item Déterminer le polynôme \( Q \) tel que \( P(x) = (x - a)Q(x) \).
\end{enumerate}

\textbf{II.} On admet que le polynôme à coefficients réels \( h(x) = \frac{1}{3}x^3 + ax^2 + bx \) vérifie l’égalité \( h(x+1) - h(x) = x^2 \).

\begin{enumerate}
    \item Sans déterminer \( a \) et \( b \), calculer \( h(-1) \), \( h(0) \) et \( h(1) \).
    \item Calculer \( a \) et \( b \).
    \item En déduire la somme \( S = 1^2 + 2^2 + 3^2 + \dots + n^2 = \sum_{k=1}^n k^2 \) sous forme factorisée.
\end{enumerate}
\end{document}