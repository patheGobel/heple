\documentclass[12pt]{article}
\usepackage{stmaryrd}
\usepackage{graphicx}
\usepackage[utf8]{inputenc}

\usepackage[french]{babel}
\usepackage[T1]{fontenc}
\usepackage{hyperref}
\usepackage{verbatim}

\usepackage{color, soul}

\usepackage{pgfplots}
\pgfplotsset{compat=1.15}
\usepackage{mathrsfs}

\usepackage{amsmath}
\usepackage{amsfonts}
\usepackage{amssymb}
\usepackage{tkz-tab}
\author{}
\title{\textbf{Correction}}
\date{\today}
\usepackage{tikz}
\usetikzlibrary{arrows, shapes.geometric, fit}

% Commande pour la couleur d'accentuation
\newcommand{\myul}[2][black]{\setulcolor{#1}\ul{#2}\setulcolor{black}}
\newcommand\tab[1][1cm]{\hspace*{#1}}

\begin{document}
\maketitle
\newpage

\section*{\underline{Exercice 1: }\textbf{3 pts}}
\subsection*{1) Mettre sous la forme } $a\sqrt{b}$ : A=$-3\sqrt{50}+2\sqrt{162}-\sqrt{200}$
\subsection*{2) Mettre sous la forme d'une fraction irrecductible } B=$\left( 2-\frac{3}{5}\right) \left(  \frac{\frac{1}{2}-\frac{3}{5}}{-\frac{4}{5}}\right) $

\section*{\underline{\textcolor{green}{Exercice 1: \textbf{3 pts}}}}

\subsection*{1) \textcolor{green}{Mettons sous la forme }} $a\sqrt{b}$ : $A = -3\sqrt{50} + 2\sqrt{162} - \sqrt{200}$

Tout d'abord, simplifions chaque terme en extrayant les racines carrées :

\[
\sqrt{50} = \sqrt{25 \cdot 2} = 5\sqrt{2}
\]
\[
\sqrt{162} = \sqrt{81 \cdot 2} = 9\sqrt{2}
\]
\[
\sqrt{200} = \sqrt{100 \cdot 2} = 10\sqrt{2}
\]

Substituons ces résultats dans l'expression initiale :

\[
A = -3(5\sqrt{2}) + 2(9\sqrt{2}) - 10\sqrt{2}
\]

Simplifions les coefficients :

\[
A = -15\sqrt{2} + 18\sqrt{2} - 10\sqrt{2}
\]

Regroupons les termes similaires :

\[
A = (-15 + 18 - 10)\sqrt{2} = -7\sqrt{2}
\]

Donc la forme simplifiée est :
\[
A = -7\sqrt{2}
\]

\subsection*{2)\textcolor{green}{Mettons sous la forme d'une fraction irréductible }} $B = \left( 2 - \frac{3}{5} \right) \left( \frac{\frac{1}{2} - \frac{3}{5}}{-\frac{4}{5}} \right)$

Simplifions d'abord l'intérieur des parenthèses :

\[
2 - \frac{3}{5} = \frac{10}{5} - \frac{3}{5} = \frac{7}{5}
\]

Ensuite, simplifions la deuxième partie à l'intérieur des parenthèses :

\[
\frac{\frac{1}{2} - \frac{3}{5}}{-\frac{4}{5}}
\]

Simplifions le numérateur :

\[
\frac{1}{2} - \frac{3}{5} = \frac{5}{10} - \frac{6}{10} = -\frac{1}{10}
\]

Donc, nous avons :

\[
\frac{-\frac{1}{10}}{-\frac{4}{5}}
\]

Simplifions cette fraction en multipliant par l'inverse :

\[
\frac{-\frac{1}{10}}{-\frac{4}{5}} = \frac{-1}{10} \times \frac{-5}{4} = \frac{1 \times 5}{10 \times 4} = \frac{5}{40} = \frac{1}{8}
\]

Maintenant, combinons ces résultats :

\[
B = \left( \frac{7}{5} \right) \left( \frac{1}{8} \right) = \frac{7 \times 1}{5 \times 8} = \frac{7}{40}
\]

Donc la forme simplifiée est :

\[
B = \frac{7}{40}
\]

Ainsi, nous avons :

1) \( A = -7\sqrt{2} \)

2) \( B = \frac{7}{40} \)


\section*{\underline{Exercice 2: }\textbf{3 pts}}
\subsection*{2) resoudre dans $\mathbb{R}^{2}$ les systèmes suivants } 
\[
\begin{cases}
2x - y = 3 \\
-x + 3y = -4
\end{cases}
\begin{cases}
x - 2y = -1 \\
-3x + 6y = 3
\end{cases}
\begin{cases}
x + y = -2 \\
xy = -15
\end{cases}
\begin{cases}
x + y = -1 \\
xy = 3
\end{cases}
\]

\section*{\underline{\textcolor{green}{Exercice 2: \textbf{3 pts}}}}

\subsection*{\textcolor{green}{2) Résoudre dans $\mathbb{R}^{2}$ les systèmes suivants }}

\[
\begin{cases}
2x - y = 3 \\
-x + 3y = -4
\end{cases}
\]

Multipliant la première équation par 3 pour obtenir les mêmes coefficients de \( y \):

\[
\begin{cases}
6x - 3y = 9 \\
-x + 3y = -4
\end{cases}
\]

Ajoutons les deux équations pour éliminer \( y \):

\[
6x - 3y - x + 3y = 9 - 4 \implies 5x = 5 \implies x = 1
\]

Substituons \( x = 1 \) dans la première équation :

\[
2(1) - y = 3 \implies 2 - y = 3 \implies y = -1
\]

La solution est donc \( (x, y) = (1, -1) \).

\[
\begin{cases}
x - 2y = -1 \\
-3x + 6y = 3
\end{cases}
\]

Observons que la deuxième équation est un multiple de la première. Divisons la deuxième équation par -3 :

\[
x - 2y = -1
\]
\[
x - 2y = -1
\]

Les deux équations sont identiques, ce qui signifie qu'il y a une infinité de solutions sur la droite \( x - 2y = -1 \). %Nous pouvons écrire la solution paramétrique :

%\[
%x = 2y - 1
%\]

%où \( y \) est un paramètre libre.

\[
\begin{cases}
x + y = -2 \\
xy = -15
\end{cases}
\]

Utilisons la première équation pour exprimer \( y \) en fonction de \( x \) :

\[
y = -2 - x
\]

Substituons dans la deuxième équation :

\[
x(-2 - x) = -15 \implies -2x - x^2 = -15 \implies x^2 + 2x - 15 = 0
\]

Résolvons cette équation quadratique en utilisant la formule quadratique \( x = \frac{-b \pm \sqrt{b^2 - 4ac}}{2a} \) :

\[
a = 1, \; b = 2, \; c = -15
\]
\[
\Delta = 2^2 - 4 \cdot 1 \cdot (-15) = 4 + 60 = 64
\]
\[
x = \frac{-2 \pm \sqrt{64}}{2} = \frac{-2 \pm 8}{2}
\]

\[
x_1 = \frac{6}{2} = 3, \quad x_2 = \frac{-10}{2} = -5
\]

Pour \( x_1 = 3 \):

\[
y = -2 - 3 = -5
\]

Pour \( x_2 = -5 \):

\[
y = -2 - (-5) = 3
\]

Les solutions sont donc \( (x, y) = (3, -5) \) et \( (x, y) = (-5, 3) \).

\[
\begin{cases}
x + y = -1 \\
xy = 3
\end{cases}
\]

Utilisons la première équation pour exprimer \( y \) en fonction de \( x \):

\[
y = -1 - x
\]

Substituons dans la deuxième équation :

\[
x(-1 - x) = 3 \implies -x - x^2 = 3 \implies x^2 + x + 3 = 0
\]

Cette équation n'a pas de solutions réelles car le discriminant est négatif :

\[
\Delta = 1^2 - 4 \cdot 1 \cdot 3 = 1 - 12 = -11
\]

Donc, il n'y a pas de solution réelle pour ce système.

Les solutions sont donc :

1. \( \begin{cases}
2x - y = 3 \\
-x + 3y = -4
\end{cases} \quad \text{Solution: } (1, -1) \)

2. \( \begin{cases}
x - 2y = -1 \\
-3x + 6y = 3
\end{cases} \quad \text{Solution: infinité de solutions sur la droite } x = 2y - 1 \)

3. \( \begin{cases}
x + y = -2 \\
xy = -15
\end{cases} \quad \text{Solutions: } (3, -5) \text{ et } (-5, 3) \)

4. \( \begin{cases}
x + y = -1 \\
xy = 3
\end{cases} \quad \text{Solution: aucune solution réelle} \)


\section*{\underline{Exercice 3: }\textbf{11 pts}}
Soit $P(x)=ax^{2}+bx+c$, donne la forme canonique de P(x) puis la forme factorisée de P(x) si $\Delta$>0 ( $2\times 1$ pt)

Donner la forme canonique de f(x) et g(x) puis la forme factorisée de h(x).
  
1) $f(x)=-3x^{2}+2x-8$     $g(x)=-4x^{2}+4x-1$    $h(x)=5x^{2}+7x-12$ \textbf{( $3\times 1$ pt)}

2)Résoudre dans $\mathbb{R}$ les équations:

a) $-3x^{2}+2x-8=0$     b)$5x^{2}+7x-12=0$    c)$-4x^{2}+4x-1=0$ \textbf{( $3\times 1$ pt)}

2)Résoudre dans $\mathbb{R}$ les inéquations suivantes:

a) $x^{2}+2x+1\leq 0$     b)$x^{2}+\sqrt{5} x+1$    c)$-x(x+3)+2x\geq -6$ \textbf{( $3\times 1$ pt)}
\section*{\underline{\textcolor{green}{Correction de l'Exercice 3: \textbf{11 pts}}}}

Soit \( P(x) = ax^{2} + bx + c \), donnons la forme canonique de \( P(x) \) puis la forme factorisée de \( P(x) \) si \( \Delta > 0 \).

\subsection*{\textcolor{green}{Forme canonique et forme factorisée}}

Pour trouver la forme canonique de \( P(x) = ax^{2} + bx + c \):

La forme canonique est \( P(x) = a\left[ \left( x + \frac{b}{2a}\right) ^{2} - \frac{\Delta}{4a^{2}}\right]  \).

Pour trouver la forme factorisée de \( P(x) \) lorsque \( \Delta > 0 \) :

La forme factorisée est \( P(x) = a\left( x + \frac{b}{2a}-\frac{\sqrt{\Delta}}{2a}\right) \left( x + \frac{b}{2a}+\frac{\sqrt{\Delta}}{2a}\right) \).

1) \( f(x) = -3x^{2} + 2x - 8 \), \( g(x) = -4x^{2} + 4x - 1 \), \( h(x) = 5x^{2} + 7x - 12 \)

\textcolor{green}{\textbf{**Forme canonique de \( f(x) \) :**}}

\[
\Delta = -92
\]

\[
f(x) = -3\left[ \left(x - \frac{1}{3}\right)^{2} + \frac{23}{9}\right] 
\]

\textcolor{green}{\textbf{**Forme canonique de \( g(x) \) :**}}

\[
\Delta = 0
\]

\[
g(x) = -4\left(x - \frac{1}{2}\right)^{2}
\]

\textcolor{green}{\textbf{**Forme factorisée de \( h(x) \) :**}}

\[
\Delta = b^2 - 4ac = 7^2 - 4(5)(-12) = 49 + 240 = 289 \quad (\Delta > 0, \text{ donc deux racines réelles})
\]
\[
x_{1,2} = \frac{-b \pm \sqrt{\Delta}}{2a} = \frac{-7 \pm \sqrt{289}}{2 \cdot 5} = \frac{-7 \pm 17}{10}
\]
\[
x_1 = 1, \quad x_2 = -\frac{12}{5}
\]
\[
h(x) = 5(x - 1)(x + \frac{12}{5})
\]

2) Résoudre dans \(\mathbb{R}\) les équations:

a) \( -3x^{2} + 2x - 8 = 0 \)

\[
\Delta = b^2 - 4ac = 2^2 - 4(-3)(-8) = 4 - 96 = -92 \quad (\Delta < 0, \text{ donc pas de solution réelle})
\]

b) \( 5x^{2} + 7x - 12 = 0 \)

\[
\Delta = b^2 - 4ac = 7^2 - 4(5)(-12) = 49 + 240 = 289 \quad (\Delta > 0, \text{ donc deux racines réelles})
\]
\[
x_{1,2} = \frac{-b \pm \sqrt{\Delta}}{2a} = \frac{-7 \pm \sqrt{289}}{2 \cdot 5} = \frac{-7 \pm 17}{10}
\]
\[
x_1 = 1, \quad x_2 = -\frac{12}{5}
\]

c) \( -4x^{2} + 4x - 1 = 0 \)

\[
\Delta = b^2 - 4ac = 4^2 - 4(-4)(-1) = 16 - 16 = 0 \quad (\Delta = 0, \text{ donc une racine double})
\]
\[
x = \frac{-b}{2a} = \frac{-4}{2(-4)} = \frac{1}{2}
\]

2) Résoudre dans \(\mathbb{R}\) les inéquations suivantes:

a) \( x^{2} + 2x + 1 \leq 0 \)

\[
x^2 + 2x + 1 = (x + 1)^2
\]
\[
(x + 1)^2 \leq 0
\]

La seule solution est \( x = -1 \).

b) \( x^{2} + \sqrt{5} x + 1 >0 \)

\[
\Delta = (\sqrt{5})^2 - 4(1)(1) = 5 - 4 = 1 \quad (\Delta > 0, \text{ donc deux racines réelles})
\]
\[
x_{1,2} = \frac{-\sqrt{5} \pm \sqrt{1}}{2(1)} = \frac{-\sqrt{5} \pm 1}{2}
\]
\[
x_1 = \frac{-\sqrt{5} - 1}{2}, \quad x_2 = \frac{-\sqrt{5} + 1}{2}
\]

La solution dans \(\mathbb{R}\) est: S= \( \left]-\infty ; \frac{-\sqrt{5} - 1}{2} \right[ \cup \left]\frac{-\sqrt{5} + 1}{2} +\infty\right[ \).

c) \( -x(x + 3) + 2x \geq -6 \)

\[
-x^2 - 3x + 2x \geq -6 \implies -x^2 - x \geq -6
\]
\[
-x^2 - x + 6 \geq 0
\]
\[
x^2 + x - 6 \leq 0
\]

Résolvons \( x^2 + x - 6 = 0 \)

\[
\Delta = 1^2 - 4(1)(-6) = 1 + 24 = 25
\]
\[
x_{1,2} = \frac{-1 \pm \sqrt{25}}{2} = \frac{-1 \pm 5}{2}
\]
\[
x_1 = 2, \quad x_2 = -3
\]

Les solutions de l'inéquation sont \( -3 \leq x \leq 2 \) c'est-à-dire $x\in\mathbb{R}\left[ -3 ; 2\right] $.

Donc S=$\left[ -3 ; 2\right]$

Ainsi, les solutions complètes sont :

1. \( f(x) = -3x^{2} + 2x - 8 \) \\
Forme canonique : \[
f(x) = -3\left[ \left(x - \frac{1}{3}\right)^{2} + \frac{23}{9}\right] 
\]\\

\( g(x) = -4x^{2} + 4x - 1 \) \\
Forme canonique : \[
g(x) = -4\left(x - \frac{1}{2}\right)^{2}
\] \\

\( h(x) = 5x^{2} + 7x - 12 \) \\
Forme factorisée : \( 5(x - 1)(x + \frac{12}{5}) \).

2. Résoudre dans \(\mathbb{R}\) les équations :

a) \( -3x^{2} + 2x - 8 = 0 \) : Pas de solution réelle.

b) \( 5x^{2} + 7x - 12 = 0 \) : Solutions : \( x_1 = 1 \), \( x_2 = -\frac{12}{5} \).

c) \( -4x^{2} + 4x - 1 = 0 \) : Solution : \( x = \frac{1}{2} \).

2. Résoudre dans \(\mathbb{R}\) les inéquations suivantes :

a) \( x^{2} + 2x + 1 \leq 0 \) : Solution : \( x = -1 \).

b) \( x^{2} + \sqrt{5} x + 1 \) : Pas de solution.

c) \( -x(x + 3) + 2x \geq -6 \) : Solutions : \( -3 \leq x \leq 2 \).

\end{document}