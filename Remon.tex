\documentclass{article}
\usepackage{amsmath}

\begin{document}

\section*{Calcul des cas positifs durant les premiers mois de la pandémie}

Lors de la première vague de la pandémie de COVID-19, 400 cas positifs ont été décomptés durant le premier mois. Chaque mois, le nombre de cas augmente de 5\%. On note \( U_{n} \) le nombre de cas positifs décomptés durant le \( n \)-ième mois de la pandémie.

\subsection*{Formule de progression géométrique}
La formule générale pour une progression géométrique est :
\[
U_{n+1} = U_n \times (1 + r)
\]
où \( r \) est le taux d'augmentation.

\subsection*{Calculs}
\begin{itemize}
    \item Données initiales :
    \[
    U_1 = 400
    \]
    \item Taux d'augmentation mensuel :
    \[
    r = 0.05
    \]
\end{itemize}

\subsubsection*{Pour \( U_2 \) (deuxième mois)}
\[
U_2 = U_1 \times (1 + r)
\]
\[
U_2 = 400 \times 1.05
\]
\[
U_2 = 420
\]

\subsubsection*{Pour \( U_3 \) (troisième mois)}
\[
U_3 = U_2 \times (1 + r)
\]
\[
U_3 = 420 \times 1.05
\]
\[
U_3 = 441
\]

\subsection*{Résultats}
Ainsi, les nombres de cas positifs décomptés durant les premiers mois de la pandémie sont :
\begin{itemize}
    \item Premier mois (\( U_1 \)) : 400 cas
    \item Deuxième mois (\( U_2 \)) : 420 cas
    \item Troisième mois (\( U_3 \)) : 441 cas
\end{itemize}

\end{document}
