\documentclass[12pt,a4paper]{article}
\usepackage{amsmath,amssymb,mathrsfs,tikz,times,pifont}
\usepackage{enumitem}
\newcommand\circitem[1]{%
\tikz[baseline=(char.base)]{
\node[circle,draw=gray, fill=red!55,
minimum size=1.2em,inner sep=0] (char) {#1};}}
\newcommand\boxitem[1]{%
\tikz[baseline=(char.base)]{
\node[fill=cyan,
minimum size=1.2em,inner sep=0] (char) {#1};}}
\setlist[enumerate,1]{label=\protect\circitem{\arabic*}}
\setlist[enumerate,2]{label=\protect\boxitem{\alph*}}
%%%::::::by chnini ameur :::::::%%%
\everymath{\displaystyle}
\usepackage[left=1cm,right=1cm,top=1cm,bottom=1.7cm]{geometry}
\usepackage{array,multirow}
\usepackage[most]{tcolorbox}
\usepackage{varwidth}
\tcbuselibrary{skins,hooks}
\usetikzlibrary{patterns}
%%%::::::by chnini ameur :::::::%%%
\newtcolorbox{exa}[2][]{enhanced,breakable,before skip=2mm,after skip=5mm,
colback=yellow!20!white,colframe=black!20!blue,boxrule=0.5mm,
attach boxed title to top left ={xshift=0.6cm,yshift*=1mm-\tcboxedtitleheight},
fonttitle=\bfseries,
title={#2},#1,
% varwidth boxed title*=-3cm,
boxed title style={frame code={
\path[fill=tcbcolback!30!black]
([yshift=-1mm,xshift=-1mm]frame.north west)
arc[start angle=0,end angle=180,radius=1mm]
([yshift=-1mm,xshift=1mm]frame.north east)
arc[start angle=180,end angle=0,radius=1mm];
\path[left color=tcbcolback!60!black,right color = tcbcolback!60!black,
middle color = tcbcolback!80!black]
([xshift=-2mm]frame.north west) -- ([xshift=2mm]frame.north east)
[rounded corners=1mm]-- ([xshift=1mm,yshift=-1mm]frame.north east)
-- (frame.south east) -- (frame.south west)
-- ([xshift=-1mm,yshift=-1mm]frame.north west)
[sharp corners]-- cycle;
},interior engine=empty,
},interior style={top color=yellow!5}}
%%%%%%%%%%%%%%%%%%%%%%%
\usepackage{fancyhdr}
\usepackage{lastpage}
\fancyhf{}
\pagestyle{fancy}
\renewcommand{\footrulewidth}{1pt}
\renewcommand{\headrulewidth}{0pt}
\renewcommand{\footruleskip}{10pt}
\fancyfoot[R]{
\color{blue}\ding{45}\ \textbf{2024}
}
\fancyfoot[L]{
\color{blue}\ding{45}\ \textbf{Prof:Mme. Mbathie}
}
\cfoot{\bf
\thepage /
\pageref{LastPage}}
\begin{document}
\renewcommand{\arraystretch}{1.5}
\renewcommand{\arrayrulewidth}{1.2pt}
\begin{tikzpicture}[overlay,remember picture]
\node[draw=blue,line width=1.2pt,fill=purple,text=blue,inner sep=3mm,rounded corners,pattern=dots]at ([yshift=-2.5cm]current page.north) {\begingroup\setlength{\fboxsep}{0pt}\colorbox{white}{\begin{tabular}{|*1{>{\centering \arraybackslash}p{0.28\textwidth}} |*2{>{\centering \arraybackslash}p{0.2\textwidth}|} *1{>{\centering \arraybackslash}p{0.19\textwidth}|} }
\hline
\multicolumn{3}{|c|}{$\diamond$$\diamond$$\diamond$\ \textbf{Lycée ST-MALÈME}\ $\diamond$$\diamond$$\diamond$ }& \textbf{A.S. : 2024/2025} \\ \hline
\textbf{Matière: Mathématiques}& \textbf{Niveau : 2}$ ^\text{\bf nd} $\textbf{S} &\textbf{Date: 03/12/2024} & \textbf{Durée : 4 heures} \\ \hline
\multicolumn{4}{|c|}{\parbox[c]{10cm}{\begin{center}
\textbf{{\Large\sffamily Devoir n$ ^{\circ} $ 1 Du 1$ ^\text{\bf er} $ Semestre}}
\end{center}}} \\ \hline
\end{tabular}}\endgroup};
\end{tikzpicture}
\vspace{3cm}
\section*{\underline{Exercice n°1 : (05pts)}}
\begin{enumerate}
\item Simplifier les nombres suivants à l'aide de puissances entières des nombres premiers :
\[
A = \dfrac{(3^4 \times 2^{-3})^3}{(9^{-1} \times 2^2)^4}, \quad B = \dfrac{9^2 \times 4^2 \times 6^2}{27^2 \times (12 \times 8)^3}.
\]

\item Montrer que $a^2 + b^2 \geq 2ab$ ; quels que soient les réels $a$, $b$.\hspace{1cm} \textbf{(1pt)}

\item Soient $a$, $b$ et $c$ des réels strictement positifs montrer que 
$(a^2 + b^2)c + (b^2 + c^2)a + (c^2 + a^2)b \geq 6abc$.\hspace{0.4cm}\textbf{(1pt)}

\item Soient $x$, $y$ et $z$ trois nombres réels, montrer que $xy + yz + xz \leq x^2 + y^2 + z^2$.\hspace{1cm}\textbf{(1pt)}
\end{enumerate}
\section*{Exercice 2 : (6pts)}
Soient $x$ et $y$ deux réels tels que $x \geq \frac{1}{2}$, $y \leq 1$ et $x - y = 3$.
\begin{enumerate}
\item Calculer $A = \sqrt{(2x - 1)^2} + \sqrt{(2y - 2)^2}$ \hspace{1cm} \textbf{(2pts)}

\item Montrer que $\frac{1}{2} \leq x \leq 4$ et $-\frac{5}{2} \leq y \leq 1$ \hspace{1cm} \textbf{(2pts)}

\item Calculer $B = |x + y - 5| + |x + y + 2|$ \hspace{1cm} \textbf{(2pts)}
\end{enumerate}
\section*{Exercice n°3 : (6pts)}

\begin{enumerate}
\item Résoudre dans $\mathbb{R}$ les équations suivantes :
    \begin{enumerate}
        \item $| -x + 4 | = 2x + 1$ \hspace{1cm} \textbf{(1.5pt)}
        \item $|5x - 2| - |3 - x| = 0$ \hspace{1cm} \textbf{(1.5pt)}
    \end{enumerate}
\item Résoudre les inéquations suivantes :
    \begin{enumerate}
        \item $|3x + 13| \leq 7$ \hspace{1cm} \textbf{(1.5pt)}
        \item $|11x - 10| > -x + 8$ \hspace{1cm} \textbf{(1.5pt)}
    \end{enumerate}
\end{enumerate}
\section*{Exercice 4 : (3pts)}

1) Calculer $(2 - \sqrt{13})^2$ et en déduire une écriture plus simple de $B = \sqrt{17 - 4\sqrt{13}}$ \hspace{8cm} \textbf{(2pts)}

2) Développer, réduire et ordonner l'expression suivante : $(2x - 3)^3 + (3x + \sqrt{2})^3$ \hspace{1cm} \textbf{(3pts)}\vspace{1cm}

\textbf{NB : La clarté et la précision de la démarche seront prises en compte}
\end{document}