\documentclass[12pt]{article}
\usepackage{stmaryrd}
\usepackage{graphicx}
\usepackage[utf8]{inputenc}

\usepackage[french]{babel}
\usepackage[T1]{fontenc}
\usepackage{hyperref}
\usepackage{verbatim}

\usepackage{color, soul}

\usepackage{pgfplots}
\pgfplotsset{compat=1.15}
\usepackage{mathrsfs}

\usepackage{amsmath}
\usepackage{amsfonts}
\usepackage{amssymb}
\usepackage{tkz-tab}
\author{Destinés à Diao}
\title{\textbf{Cramer}}
\date{\today}
\usepackage{tikz}
\usetikzlibrary{arrows, shapes.geometric, fit}

% Commande pour la couleur d'accentuation
\newcommand{\myul}[2][black]{\setulcolor{#1}\ul{#2}\setulcolor{black}}
\newcommand\tab[1][1cm]{\hspace*{#1}}

\begin{document}
\maketitle
\newpage

\section*{Nous devons résoudre le système suivant en utilisant la méthode de Cramer :}

\[
\begin{cases}
-3x + 3y = 1 \\
2x - 6y = -2
\end{cases}
\]


\[
A = \begin{pmatrix}
-3 & 3 \\
2 & -6
\end{pmatrix}, \quad
\mathbf{b} = \begin{pmatrix}
1 \\
-2
\end{pmatrix}
\]

Ensuite, calculons le déterminant de la matrice $A$ :

\[
\det(A) = \begin{vmatrix}
-3 & 3 \\
2 & -6
\end{vmatrix} = (-3)(-6) - (3)(2) = 18 - 6 = 12
\]

Calculons maintenant le déterminant de $A_x$, où la première colonne de $A$ est remplacée par $\mathbf{b}$ :

\[
A_x = \begin{pmatrix}
1 & 3 \\
-2 & -6
\end{pmatrix}
\]

\[
\det(A_x) = \begin{vmatrix}
1 & 3 \\
-2 & -6
\end{vmatrix} = (1)(-6) - (3)(-2) = -6 + 6 = 0
\]

Calculons ensuite le déterminant de $A_y$, où la deuxième colonne de $A$ est remplacée par $\mathbf{b}$ :

\[
A_y = \begin{pmatrix}
-3 & 1 \\
2 & -2
\end{pmatrix}
\]

\[
\det(A_y) = \begin{vmatrix}
-3 & 1 \\
2 & -2
\end{vmatrix} = (-3)(-2) - (1)(2) = 6 - 2 = 4
\]

Selon la règle de Cramer, les solutions $x$ et $y$ sont données par :

\[
x = \frac{\det(A_x)}{\det(A)} = \frac{0}{12} = 0
\]

\[
y = \frac{\det(A_y)}{\det(A)} = \frac{4}{12} = \frac{1}{3}
\]

Ainsi, la solution du système est :

\[
\boxed{x = 0, \; y = \frac{1}{3}}
\]

\section*{Nous devons résoudre le système suivant en utilisant la méthode de Cramer :} 

\[
\begin{cases}
3x + y + 7 = 0 \\
-x + 2y = 9
\end{cases}
\]

Réécrivons le système sous forme standard :

\[
\begin{cases}
3x + y = -7 \\
-x + 2y = 9
\end{cases}
\]

\[
A = \begin{pmatrix}
3 & 1 \\
-1 & 2
\end{pmatrix}, \quad
\mathbf{b} = \begin{pmatrix}
-7 \\
9
\end{pmatrix}
\]

Ensuite, calculons le déterminant de $A$ :

\[
\det(A) = \begin{vmatrix}
3 & 1 \\
-1 & 2
\end{vmatrix} = (3)(2) - (1)(-1) = 6 + 1 = 7
\]

Calculons maintenant le déterminant de $A_x$, où la première colonne de $A$ est remplacée par $\mathbf{b}$ :

\[
A_x = \begin{pmatrix}
-7 & 1 \\
9 & 2
\end{pmatrix}
\]

\[
\det(A_x) = \begin{vmatrix}
-7 & 1 \\
9 & 2
\end{vmatrix} = (-7)(2) - (1)(9) = -14 - 9 = -23
\]

Calculons ensuite le déterminant de $A_y$, où la deuxième colonne de $A$ est remplacée par $\mathbf{b}$ :

\[
A_y = \begin{pmatrix}
3 & -7 \\
-1 & 9
\end{pmatrix}
\]

\[
\det(A_y) = \begin{vmatrix}
3 & -7 \\
-1 & 9
\end{vmatrix} = (3)(9) - (-7)(-1) = 27 - 7 = 20
\]

Selon la règle de Cramer, les solutions $x$ et $y$ sont données par :

\[
x = \frac{\det(A_x)}{\det(A)} = \frac{-23}{7}
\]

\[
y = \frac{\det(A_y)}{\det(A)} = \frac{20}{7}
\]

Ainsi, la solution du système est :

\[
\boxed{x = -\frac{23}{7}, \; y = \frac{20}{7}}
\]

\section*{Nous devons résoudre le système suivant en utilisant la méthode de Cramer :}

\[
\begin{cases}
x + y + 4 = 0 \\
-3x + 3y = 2
\end{cases}
\]

Réécrivons le système sous forme standard :

\[
\begin{cases}
x + y = -4 \\
-3x + 3y = 2
\end{cases}
\]

\[
A = \begin{pmatrix}
1 & 1 \\
-3 & 3
\end{pmatrix}, \quad
\mathbf{b} = \begin{pmatrix}
-4 \\
2
\end{pmatrix}
\]

Ensuite, calculons le déterminant de $A$ :

\[
\det(A) = \begin{vmatrix}
1 & 1 \\
-3 & 3
\end{vmatrix} = (1)(3) - (1)(-3) = 3 + 3 = 6
\]

Calculons maintenant le déterminant de $A_x$, où la première colonne de $A$ est remplacée par $\mathbf{b}$ :

\[
A_x = \begin{pmatrix}
-4 & 1 \\
2 & 3
\end{pmatrix}
\]

\[
\det(A_x) = \begin{vmatrix}
-4 & 1 \\
2 & 3
\end{vmatrix} = (-4)(3) - (1)(2) = -12 - 2 = -14
\]

Calculons ensuite le déterminant de $A_y$, où la deuxième colonne de $A$ est remplacée par $\mathbf{b}$ :

\[
A_y = \begin{pmatrix}
1 & -4 \\
-3 & 2
\end{pmatrix}
\]

\[
\det(A_y) = \begin{vmatrix}
1 & -4 \\
-3 & 2
\end{vmatrix} = (1)(2) - (-4)(-3) = 2 - 12 = -10
\]

Selon la règle de Cramer, les solutions $x$ et $y$ sont données par :

\[
x = \frac{\det(A_x)}{\det(A)} = \frac{-14}{6} = -\frac{7}{3}
\]

\[
y = \frac{\det(A_y)}{\det(A)} = \frac{-10}{6} = -\frac{5}{3}
\]

Ainsi, la solution du système est :

\[
\boxed{x = -\frac{7}{3}, \; y = -\frac{5}{3}}
\]

Nous devons résoudre le système suivant en utilisant la méthode de Cramer :

\[
\begin{cases}
3x - y = -5 \\
-6x + 2y = 10
\end{cases}
\]

Tout d'abord, écrivons le système sous forme :

\[
A = \begin{pmatrix}
3 & -1 \\
-6 & 2
\end{pmatrix}, \quad
\mathbf{b} = \begin{pmatrix}
-5 \\
10
\end{pmatrix}
\]

Ensuite, calculons le déterminant de $A$ :

\[
\det(A) = \begin{vmatrix}
3 & -1 \\
-6 & 2
\end{vmatrix} = (3)(2) - (-1)(-6) = 6 - 6 = 0
\]

Puisque le déterminant de $A$ est zéro, cela signifie que le système d'équations est soit dépendant (ayant une infinité de solutions) soit inconsistant (n'ayant aucune solution). Pour déterminer laquelle des situations s'applique, nous devons examiner les équations plus en détail.

Nous observons que la seconde équation est une multiple de la première. En fait, si nous multiplions la première équation par 2, nous obtenons :

\[
2(3x - y) = 2(-5)
\]

\[
6x - 2y = -10
\]

qui est l'équation opposée de la seconde équation \(-6x + 2y = 10\). Cela signifie que les deux équations sont contradictoires, donc le système est inconsistant et n'a pas de solution.

Ainsi, la solution du système est :

\[
\boxed{\text{Pas de solution}}
\]

\end{document}